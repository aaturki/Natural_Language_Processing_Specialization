\documentclass[11pt]{article}

    \usepackage[breakable]{tcolorbox}
    \usepackage{parskip} % Stop auto-indenting (to mimic markdown behaviour)
    
    \usepackage{iftex}
    \ifPDFTeX
    	\usepackage[T1]{fontenc}
    	\usepackage{mathpazo}
    \else
    	\usepackage{fontspec}
    \fi

    % Basic figure setup, for now with no caption control since it's done
    % automatically by Pandoc (which extracts ![](path) syntax from Markdown).
    \usepackage{graphicx}
    % Maintain compatibility with old templates. Remove in nbconvert 6.0
    \let\Oldincludegraphics\includegraphics
    % Ensure that by default, figures have no caption (until we provide a
    % proper Figure object with a Caption API and a way to capture that
    % in the conversion process - todo).
    \usepackage{caption}
    \DeclareCaptionFormat{nocaption}{}
    \captionsetup{format=nocaption,aboveskip=0pt,belowskip=0pt}

    \usepackage[Export]{adjustbox} % Used to constrain images to a maximum size
    \adjustboxset{max size={0.9\linewidth}{0.9\paperheight}}
    \usepackage{float}
    \floatplacement{figure}{H} % forces figures to be placed at the correct location
    \usepackage{xcolor} % Allow colors to be defined
    \usepackage{enumerate} % Needed for markdown enumerations to work
    \usepackage{geometry} % Used to adjust the document margins
    \usepackage{amsmath} % Equations
    \usepackage{amssymb} % Equations
    \usepackage{textcomp} % defines textquotesingle
    % Hack from http://tex.stackexchange.com/a/47451/13684:
    \AtBeginDocument{%
        \def\PYZsq{\textquotesingle}% Upright quotes in Pygmentized code
    }
    \usepackage{upquote} % Upright quotes for verbatim code
    \usepackage{eurosym} % defines \euro
    \usepackage[mathletters]{ucs} % Extended unicode (utf-8) support
    \usepackage{fancyvrb} % verbatim replacement that allows latex
    \usepackage{grffile} % extends the file name processing of package graphics 
                         % to support a larger range
    \makeatletter % fix for grffile with XeLaTeX
    \def\Gread@@xetex#1{%
      \IfFileExists{"\Gin@base".bb}%
      {\Gread@eps{\Gin@base.bb}}%
      {\Gread@@xetex@aux#1}%
    }
    \makeatother

    % The hyperref package gives us a pdf with properly built
    % internal navigation ('pdf bookmarks' for the table of contents,
    % internal cross-reference links, web links for URLs, etc.)
    \usepackage{hyperref}
    % The default LaTeX title has an obnoxious amount of whitespace. By default,
    % titling removes some of it. It also provides customization options.
    \usepackage{titling}
    \usepackage{longtable} % longtable support required by pandoc >1.10
    \usepackage{booktabs}  % table support for pandoc > 1.12.2
    \usepackage[inline]{enumitem} % IRkernel/repr support (it uses the enumerate* environment)
    \usepackage[normalem]{ulem} % ulem is needed to support strikethroughs (\sout)
                                % normalem makes italics be italics, not underlines
    \usepackage{mathrsfs}
    

    
    % Colors for the hyperref package
    \definecolor{urlcolor}{rgb}{0,.145,.698}
    \definecolor{linkcolor}{rgb}{.71,0.21,0.01}
    \definecolor{citecolor}{rgb}{.12,.54,.11}

    % ANSI colors
    \definecolor{ansi-black}{HTML}{3E424D}
    \definecolor{ansi-black-intense}{HTML}{282C36}
    \definecolor{ansi-red}{HTML}{E75C58}
    \definecolor{ansi-red-intense}{HTML}{B22B31}
    \definecolor{ansi-green}{HTML}{00A250}
    \definecolor{ansi-green-intense}{HTML}{007427}
    \definecolor{ansi-yellow}{HTML}{DDB62B}
    \definecolor{ansi-yellow-intense}{HTML}{B27D12}
    \definecolor{ansi-blue}{HTML}{208FFB}
    \definecolor{ansi-blue-intense}{HTML}{0065CA}
    \definecolor{ansi-magenta}{HTML}{D160C4}
    \definecolor{ansi-magenta-intense}{HTML}{A03196}
    \definecolor{ansi-cyan}{HTML}{60C6C8}
    \definecolor{ansi-cyan-intense}{HTML}{258F8F}
    \definecolor{ansi-white}{HTML}{C5C1B4}
    \definecolor{ansi-white-intense}{HTML}{A1A6B2}
    \definecolor{ansi-default-inverse-fg}{HTML}{FFFFFF}
    \definecolor{ansi-default-inverse-bg}{HTML}{000000}

    % commands and environments needed by pandoc snippets
    % extracted from the output of `pandoc -s`
    \providecommand{\tightlist}{%
      \setlength{\itemsep}{0pt}\setlength{\parskip}{0pt}}
    \DefineVerbatimEnvironment{Highlighting}{Verbatim}{commandchars=\\\{\}}
    % Add ',fontsize=\small' for more characters per line
    \newenvironment{Shaded}{}{}
    \newcommand{\KeywordTok}[1]{\textcolor[rgb]{0.00,0.44,0.13}{\textbf{{#1}}}}
    \newcommand{\DataTypeTok}[1]{\textcolor[rgb]{0.56,0.13,0.00}{{#1}}}
    \newcommand{\DecValTok}[1]{\textcolor[rgb]{0.25,0.63,0.44}{{#1}}}
    \newcommand{\BaseNTok}[1]{\textcolor[rgb]{0.25,0.63,0.44}{{#1}}}
    \newcommand{\FloatTok}[1]{\textcolor[rgb]{0.25,0.63,0.44}{{#1}}}
    \newcommand{\CharTok}[1]{\textcolor[rgb]{0.25,0.44,0.63}{{#1}}}
    \newcommand{\StringTok}[1]{\textcolor[rgb]{0.25,0.44,0.63}{{#1}}}
    \newcommand{\CommentTok}[1]{\textcolor[rgb]{0.38,0.63,0.69}{\textit{{#1}}}}
    \newcommand{\OtherTok}[1]{\textcolor[rgb]{0.00,0.44,0.13}{{#1}}}
    \newcommand{\AlertTok}[1]{\textcolor[rgb]{1.00,0.00,0.00}{\textbf{{#1}}}}
    \newcommand{\FunctionTok}[1]{\textcolor[rgb]{0.02,0.16,0.49}{{#1}}}
    \newcommand{\RegionMarkerTok}[1]{{#1}}
    \newcommand{\ErrorTok}[1]{\textcolor[rgb]{1.00,0.00,0.00}{\textbf{{#1}}}}
    \newcommand{\NormalTok}[1]{{#1}}
    
    % Additional commands for more recent versions of Pandoc
    \newcommand{\ConstantTok}[1]{\textcolor[rgb]{0.53,0.00,0.00}{{#1}}}
    \newcommand{\SpecialCharTok}[1]{\textcolor[rgb]{0.25,0.44,0.63}{{#1}}}
    \newcommand{\VerbatimStringTok}[1]{\textcolor[rgb]{0.25,0.44,0.63}{{#1}}}
    \newcommand{\SpecialStringTok}[1]{\textcolor[rgb]{0.73,0.40,0.53}{{#1}}}
    \newcommand{\ImportTok}[1]{{#1}}
    \newcommand{\DocumentationTok}[1]{\textcolor[rgb]{0.73,0.13,0.13}{\textit{{#1}}}}
    \newcommand{\AnnotationTok}[1]{\textcolor[rgb]{0.38,0.63,0.69}{\textbf{\textit{{#1}}}}}
    \newcommand{\CommentVarTok}[1]{\textcolor[rgb]{0.38,0.63,0.69}{\textbf{\textit{{#1}}}}}
    \newcommand{\VariableTok}[1]{\textcolor[rgb]{0.10,0.09,0.49}{{#1}}}
    \newcommand{\ControlFlowTok}[1]{\textcolor[rgb]{0.00,0.44,0.13}{\textbf{{#1}}}}
    \newcommand{\OperatorTok}[1]{\textcolor[rgb]{0.40,0.40,0.40}{{#1}}}
    \newcommand{\BuiltInTok}[1]{{#1}}
    \newcommand{\ExtensionTok}[1]{{#1}}
    \newcommand{\PreprocessorTok}[1]{\textcolor[rgb]{0.74,0.48,0.00}{{#1}}}
    \newcommand{\AttributeTok}[1]{\textcolor[rgb]{0.49,0.56,0.16}{{#1}}}
    \newcommand{\InformationTok}[1]{\textcolor[rgb]{0.38,0.63,0.69}{\textbf{\textit{{#1}}}}}
    \newcommand{\WarningTok}[1]{\textcolor[rgb]{0.38,0.63,0.69}{\textbf{\textit{{#1}}}}}
    
    
    % Define a nice break command that doesn't care if a line doesn't already
    % exist.
    \def\br{\hspace*{\fill} \\* }
    % Math Jax compatibility definitions
    \def\gt{>}
    \def\lt{<}
    \let\Oldtex\TeX
    \let\Oldlatex\LaTeX
    \renewcommand{\TeX}{\textrm{\Oldtex}}
    \renewcommand{\LaTeX}{\textrm{\Oldlatex}}
    % Document parameters
    % Document title
    \title{C1\_W2\_Assignment}
    
    
    
    
    
% Pygments definitions
\makeatletter
\def\PY@reset{\let\PY@it=\relax \let\PY@bf=\relax%
    \let\PY@ul=\relax \let\PY@tc=\relax%
    \let\PY@bc=\relax \let\PY@ff=\relax}
\def\PY@tok#1{\csname PY@tok@#1\endcsname}
\def\PY@toks#1+{\ifx\relax#1\empty\else%
    \PY@tok{#1}\expandafter\PY@toks\fi}
\def\PY@do#1{\PY@bc{\PY@tc{\PY@ul{%
    \PY@it{\PY@bf{\PY@ff{#1}}}}}}}
\def\PY#1#2{\PY@reset\PY@toks#1+\relax+\PY@do{#2}}

\expandafter\def\csname PY@tok@w\endcsname{\def\PY@tc##1{\textcolor[rgb]{0.73,0.73,0.73}{##1}}}
\expandafter\def\csname PY@tok@c\endcsname{\let\PY@it=\textit\def\PY@tc##1{\textcolor[rgb]{0.25,0.50,0.50}{##1}}}
\expandafter\def\csname PY@tok@cp\endcsname{\def\PY@tc##1{\textcolor[rgb]{0.74,0.48,0.00}{##1}}}
\expandafter\def\csname PY@tok@k\endcsname{\let\PY@bf=\textbf\def\PY@tc##1{\textcolor[rgb]{0.00,0.50,0.00}{##1}}}
\expandafter\def\csname PY@tok@kp\endcsname{\def\PY@tc##1{\textcolor[rgb]{0.00,0.50,0.00}{##1}}}
\expandafter\def\csname PY@tok@kt\endcsname{\def\PY@tc##1{\textcolor[rgb]{0.69,0.00,0.25}{##1}}}
\expandafter\def\csname PY@tok@o\endcsname{\def\PY@tc##1{\textcolor[rgb]{0.40,0.40,0.40}{##1}}}
\expandafter\def\csname PY@tok@ow\endcsname{\let\PY@bf=\textbf\def\PY@tc##1{\textcolor[rgb]{0.67,0.13,1.00}{##1}}}
\expandafter\def\csname PY@tok@nb\endcsname{\def\PY@tc##1{\textcolor[rgb]{0.00,0.50,0.00}{##1}}}
\expandafter\def\csname PY@tok@nf\endcsname{\def\PY@tc##1{\textcolor[rgb]{0.00,0.00,1.00}{##1}}}
\expandafter\def\csname PY@tok@nc\endcsname{\let\PY@bf=\textbf\def\PY@tc##1{\textcolor[rgb]{0.00,0.00,1.00}{##1}}}
\expandafter\def\csname PY@tok@nn\endcsname{\let\PY@bf=\textbf\def\PY@tc##1{\textcolor[rgb]{0.00,0.00,1.00}{##1}}}
\expandafter\def\csname PY@tok@ne\endcsname{\let\PY@bf=\textbf\def\PY@tc##1{\textcolor[rgb]{0.82,0.25,0.23}{##1}}}
\expandafter\def\csname PY@tok@nv\endcsname{\def\PY@tc##1{\textcolor[rgb]{0.10,0.09,0.49}{##1}}}
\expandafter\def\csname PY@tok@no\endcsname{\def\PY@tc##1{\textcolor[rgb]{0.53,0.00,0.00}{##1}}}
\expandafter\def\csname PY@tok@nl\endcsname{\def\PY@tc##1{\textcolor[rgb]{0.63,0.63,0.00}{##1}}}
\expandafter\def\csname PY@tok@ni\endcsname{\let\PY@bf=\textbf\def\PY@tc##1{\textcolor[rgb]{0.60,0.60,0.60}{##1}}}
\expandafter\def\csname PY@tok@na\endcsname{\def\PY@tc##1{\textcolor[rgb]{0.49,0.56,0.16}{##1}}}
\expandafter\def\csname PY@tok@nt\endcsname{\let\PY@bf=\textbf\def\PY@tc##1{\textcolor[rgb]{0.00,0.50,0.00}{##1}}}
\expandafter\def\csname PY@tok@nd\endcsname{\def\PY@tc##1{\textcolor[rgb]{0.67,0.13,1.00}{##1}}}
\expandafter\def\csname PY@tok@s\endcsname{\def\PY@tc##1{\textcolor[rgb]{0.73,0.13,0.13}{##1}}}
\expandafter\def\csname PY@tok@sd\endcsname{\let\PY@it=\textit\def\PY@tc##1{\textcolor[rgb]{0.73,0.13,0.13}{##1}}}
\expandafter\def\csname PY@tok@si\endcsname{\let\PY@bf=\textbf\def\PY@tc##1{\textcolor[rgb]{0.73,0.40,0.53}{##1}}}
\expandafter\def\csname PY@tok@se\endcsname{\let\PY@bf=\textbf\def\PY@tc##1{\textcolor[rgb]{0.73,0.40,0.13}{##1}}}
\expandafter\def\csname PY@tok@sr\endcsname{\def\PY@tc##1{\textcolor[rgb]{0.73,0.40,0.53}{##1}}}
\expandafter\def\csname PY@tok@ss\endcsname{\def\PY@tc##1{\textcolor[rgb]{0.10,0.09,0.49}{##1}}}
\expandafter\def\csname PY@tok@sx\endcsname{\def\PY@tc##1{\textcolor[rgb]{0.00,0.50,0.00}{##1}}}
\expandafter\def\csname PY@tok@m\endcsname{\def\PY@tc##1{\textcolor[rgb]{0.40,0.40,0.40}{##1}}}
\expandafter\def\csname PY@tok@gh\endcsname{\let\PY@bf=\textbf\def\PY@tc##1{\textcolor[rgb]{0.00,0.00,0.50}{##1}}}
\expandafter\def\csname PY@tok@gu\endcsname{\let\PY@bf=\textbf\def\PY@tc##1{\textcolor[rgb]{0.50,0.00,0.50}{##1}}}
\expandafter\def\csname PY@tok@gd\endcsname{\def\PY@tc##1{\textcolor[rgb]{0.63,0.00,0.00}{##1}}}
\expandafter\def\csname PY@tok@gi\endcsname{\def\PY@tc##1{\textcolor[rgb]{0.00,0.63,0.00}{##1}}}
\expandafter\def\csname PY@tok@gr\endcsname{\def\PY@tc##1{\textcolor[rgb]{1.00,0.00,0.00}{##1}}}
\expandafter\def\csname PY@tok@ge\endcsname{\let\PY@it=\textit}
\expandafter\def\csname PY@tok@gs\endcsname{\let\PY@bf=\textbf}
\expandafter\def\csname PY@tok@gp\endcsname{\let\PY@bf=\textbf\def\PY@tc##1{\textcolor[rgb]{0.00,0.00,0.50}{##1}}}
\expandafter\def\csname PY@tok@go\endcsname{\def\PY@tc##1{\textcolor[rgb]{0.53,0.53,0.53}{##1}}}
\expandafter\def\csname PY@tok@gt\endcsname{\def\PY@tc##1{\textcolor[rgb]{0.00,0.27,0.87}{##1}}}
\expandafter\def\csname PY@tok@err\endcsname{\def\PY@bc##1{\setlength{\fboxsep}{0pt}\fcolorbox[rgb]{1.00,0.00,0.00}{1,1,1}{\strut ##1}}}
\expandafter\def\csname PY@tok@kc\endcsname{\let\PY@bf=\textbf\def\PY@tc##1{\textcolor[rgb]{0.00,0.50,0.00}{##1}}}
\expandafter\def\csname PY@tok@kd\endcsname{\let\PY@bf=\textbf\def\PY@tc##1{\textcolor[rgb]{0.00,0.50,0.00}{##1}}}
\expandafter\def\csname PY@tok@kn\endcsname{\let\PY@bf=\textbf\def\PY@tc##1{\textcolor[rgb]{0.00,0.50,0.00}{##1}}}
\expandafter\def\csname PY@tok@kr\endcsname{\let\PY@bf=\textbf\def\PY@tc##1{\textcolor[rgb]{0.00,0.50,0.00}{##1}}}
\expandafter\def\csname PY@tok@bp\endcsname{\def\PY@tc##1{\textcolor[rgb]{0.00,0.50,0.00}{##1}}}
\expandafter\def\csname PY@tok@fm\endcsname{\def\PY@tc##1{\textcolor[rgb]{0.00,0.00,1.00}{##1}}}
\expandafter\def\csname PY@tok@vc\endcsname{\def\PY@tc##1{\textcolor[rgb]{0.10,0.09,0.49}{##1}}}
\expandafter\def\csname PY@tok@vg\endcsname{\def\PY@tc##1{\textcolor[rgb]{0.10,0.09,0.49}{##1}}}
\expandafter\def\csname PY@tok@vi\endcsname{\def\PY@tc##1{\textcolor[rgb]{0.10,0.09,0.49}{##1}}}
\expandafter\def\csname PY@tok@vm\endcsname{\def\PY@tc##1{\textcolor[rgb]{0.10,0.09,0.49}{##1}}}
\expandafter\def\csname PY@tok@sa\endcsname{\def\PY@tc##1{\textcolor[rgb]{0.73,0.13,0.13}{##1}}}
\expandafter\def\csname PY@tok@sb\endcsname{\def\PY@tc##1{\textcolor[rgb]{0.73,0.13,0.13}{##1}}}
\expandafter\def\csname PY@tok@sc\endcsname{\def\PY@tc##1{\textcolor[rgb]{0.73,0.13,0.13}{##1}}}
\expandafter\def\csname PY@tok@dl\endcsname{\def\PY@tc##1{\textcolor[rgb]{0.73,0.13,0.13}{##1}}}
\expandafter\def\csname PY@tok@s2\endcsname{\def\PY@tc##1{\textcolor[rgb]{0.73,0.13,0.13}{##1}}}
\expandafter\def\csname PY@tok@sh\endcsname{\def\PY@tc##1{\textcolor[rgb]{0.73,0.13,0.13}{##1}}}
\expandafter\def\csname PY@tok@s1\endcsname{\def\PY@tc##1{\textcolor[rgb]{0.73,0.13,0.13}{##1}}}
\expandafter\def\csname PY@tok@mb\endcsname{\def\PY@tc##1{\textcolor[rgb]{0.40,0.40,0.40}{##1}}}
\expandafter\def\csname PY@tok@mf\endcsname{\def\PY@tc##1{\textcolor[rgb]{0.40,0.40,0.40}{##1}}}
\expandafter\def\csname PY@tok@mh\endcsname{\def\PY@tc##1{\textcolor[rgb]{0.40,0.40,0.40}{##1}}}
\expandafter\def\csname PY@tok@mi\endcsname{\def\PY@tc##1{\textcolor[rgb]{0.40,0.40,0.40}{##1}}}
\expandafter\def\csname PY@tok@il\endcsname{\def\PY@tc##1{\textcolor[rgb]{0.40,0.40,0.40}{##1}}}
\expandafter\def\csname PY@tok@mo\endcsname{\def\PY@tc##1{\textcolor[rgb]{0.40,0.40,0.40}{##1}}}
\expandafter\def\csname PY@tok@ch\endcsname{\let\PY@it=\textit\def\PY@tc##1{\textcolor[rgb]{0.25,0.50,0.50}{##1}}}
\expandafter\def\csname PY@tok@cm\endcsname{\let\PY@it=\textit\def\PY@tc##1{\textcolor[rgb]{0.25,0.50,0.50}{##1}}}
\expandafter\def\csname PY@tok@cpf\endcsname{\let\PY@it=\textit\def\PY@tc##1{\textcolor[rgb]{0.25,0.50,0.50}{##1}}}
\expandafter\def\csname PY@tok@c1\endcsname{\let\PY@it=\textit\def\PY@tc##1{\textcolor[rgb]{0.25,0.50,0.50}{##1}}}
\expandafter\def\csname PY@tok@cs\endcsname{\let\PY@it=\textit\def\PY@tc##1{\textcolor[rgb]{0.25,0.50,0.50}{##1}}}

\def\PYZbs{\char`\\}
\def\PYZus{\char`\_}
\def\PYZob{\char`\{}
\def\PYZcb{\char`\}}
\def\PYZca{\char`\^}
\def\PYZam{\char`\&}
\def\PYZlt{\char`\<}
\def\PYZgt{\char`\>}
\def\PYZsh{\char`\#}
\def\PYZpc{\char`\%}
\def\PYZdl{\char`\$}
\def\PYZhy{\char`\-}
\def\PYZsq{\char`\'}
\def\PYZdq{\char`\"}
\def\PYZti{\char`\~}
% for compatibility with earlier versions
\def\PYZat{@}
\def\PYZlb{[}
\def\PYZrb{]}
\makeatother


    % For linebreaks inside Verbatim environment from package fancyvrb. 
    \makeatletter
        \newbox\Wrappedcontinuationbox 
        \newbox\Wrappedvisiblespacebox 
        \newcommand*\Wrappedvisiblespace {\textcolor{red}{\textvisiblespace}} 
        \newcommand*\Wrappedcontinuationsymbol {\textcolor{red}{\llap{\tiny$\m@th\hookrightarrow$}}} 
        \newcommand*\Wrappedcontinuationindent {3ex } 
        \newcommand*\Wrappedafterbreak {\kern\Wrappedcontinuationindent\copy\Wrappedcontinuationbox} 
        % Take advantage of the already applied Pygments mark-up to insert 
        % potential linebreaks for TeX processing. 
        %        {, <, #, %, $, ' and ": go to next line. 
        %        _, }, ^, &, >, - and ~: stay at end of broken line. 
        % Use of \textquotesingle for straight quote. 
        \newcommand*\Wrappedbreaksatspecials {% 
            \def\PYGZus{\discretionary{\char`\_}{\Wrappedafterbreak}{\char`\_}}% 
            \def\PYGZob{\discretionary{}{\Wrappedafterbreak\char`\{}{\char`\{}}% 
            \def\PYGZcb{\discretionary{\char`\}}{\Wrappedafterbreak}{\char`\}}}% 
            \def\PYGZca{\discretionary{\char`\^}{\Wrappedafterbreak}{\char`\^}}% 
            \def\PYGZam{\discretionary{\char`\&}{\Wrappedafterbreak}{\char`\&}}% 
            \def\PYGZlt{\discretionary{}{\Wrappedafterbreak\char`\<}{\char`\<}}% 
            \def\PYGZgt{\discretionary{\char`\>}{\Wrappedafterbreak}{\char`\>}}% 
            \def\PYGZsh{\discretionary{}{\Wrappedafterbreak\char`\#}{\char`\#}}% 
            \def\PYGZpc{\discretionary{}{\Wrappedafterbreak\char`\%}{\char`\%}}% 
            \def\PYGZdl{\discretionary{}{\Wrappedafterbreak\char`\$}{\char`\$}}% 
            \def\PYGZhy{\discretionary{\char`\-}{\Wrappedafterbreak}{\char`\-}}% 
            \def\PYGZsq{\discretionary{}{\Wrappedafterbreak\textquotesingle}{\textquotesingle}}% 
            \def\PYGZdq{\discretionary{}{\Wrappedafterbreak\char`\"}{\char`\"}}% 
            \def\PYGZti{\discretionary{\char`\~}{\Wrappedafterbreak}{\char`\~}}% 
        } 
        % Some characters . , ; ? ! / are not pygmentized. 
        % This macro makes them "active" and they will insert potential linebreaks 
        \newcommand*\Wrappedbreaksatpunct {% 
            \lccode`\~`\.\lowercase{\def~}{\discretionary{\hbox{\char`\.}}{\Wrappedafterbreak}{\hbox{\char`\.}}}% 
            \lccode`\~`\,\lowercase{\def~}{\discretionary{\hbox{\char`\,}}{\Wrappedafterbreak}{\hbox{\char`\,}}}% 
            \lccode`\~`\;\lowercase{\def~}{\discretionary{\hbox{\char`\;}}{\Wrappedafterbreak}{\hbox{\char`\;}}}% 
            \lccode`\~`\:\lowercase{\def~}{\discretionary{\hbox{\char`\:}}{\Wrappedafterbreak}{\hbox{\char`\:}}}% 
            \lccode`\~`\?\lowercase{\def~}{\discretionary{\hbox{\char`\?}}{\Wrappedafterbreak}{\hbox{\char`\?}}}% 
            \lccode`\~`\!\lowercase{\def~}{\discretionary{\hbox{\char`\!}}{\Wrappedafterbreak}{\hbox{\char`\!}}}% 
            \lccode`\~`\/\lowercase{\def~}{\discretionary{\hbox{\char`\/}}{\Wrappedafterbreak}{\hbox{\char`\/}}}% 
            \catcode`\.\active
            \catcode`\,\active 
            \catcode`\;\active
            \catcode`\:\active
            \catcode`\?\active
            \catcode`\!\active
            \catcode`\/\active 
            \lccode`\~`\~ 	
        }
    \makeatother

    \let\OriginalVerbatim=\Verbatim
    \makeatletter
    \renewcommand{\Verbatim}[1][1]{%
        %\parskip\z@skip
        \sbox\Wrappedcontinuationbox {\Wrappedcontinuationsymbol}%
        \sbox\Wrappedvisiblespacebox {\FV@SetupFont\Wrappedvisiblespace}%
        \def\FancyVerbFormatLine ##1{\hsize\linewidth
            \vtop{\raggedright\hyphenpenalty\z@\exhyphenpenalty\z@
                \doublehyphendemerits\z@\finalhyphendemerits\z@
                \strut ##1\strut}%
        }%
        % If the linebreak is at a space, the latter will be displayed as visible
        % space at end of first line, and a continuation symbol starts next line.
        % Stretch/shrink are however usually zero for typewriter font.
        \def\FV@Space {%
            \nobreak\hskip\z@ plus\fontdimen3\font minus\fontdimen4\font
            \discretionary{\copy\Wrappedvisiblespacebox}{\Wrappedafterbreak}
            {\kern\fontdimen2\font}%
        }%
        
        % Allow breaks at special characters using \PYG... macros.
        \Wrappedbreaksatspecials
        % Breaks at punctuation characters . , ; ? ! and / need catcode=\active 	
        \OriginalVerbatim[#1,codes*=\Wrappedbreaksatpunct]%
    }
    \makeatother

    % Exact colors from NB
    \definecolor{incolor}{HTML}{303F9F}
    \definecolor{outcolor}{HTML}{D84315}
    \definecolor{cellborder}{HTML}{CFCFCF}
    \definecolor{cellbackground}{HTML}{F7F7F7}
    
    % prompt
    \makeatletter
    \newcommand{\boxspacing}{\kern\kvtcb@left@rule\kern\kvtcb@boxsep}
    \makeatother
    \newcommand{\prompt}[4]{
        \ttfamily\llap{{\color{#2}[#3]:\hspace{3pt}#4}}\vspace{-\baselineskip}
    }
    

    
    % Prevent overflowing lines due to hard-to-break entities
    \sloppy 
    % Setup hyperref package
    \hypersetup{
      breaklinks=true,  % so long urls are correctly broken across lines
      colorlinks=true,
      urlcolor=urlcolor,
      linkcolor=linkcolor,
      citecolor=citecolor,
      }
    % Slightly bigger margins than the latex defaults
    
    \geometry{verbose,tmargin=1in,bmargin=1in,lmargin=1in,rmargin=1in}
    
    

\begin{document}
    
    \maketitle
    
    

    
    \hypertarget{assignment-2-naive-bayes}{%
\section{Assignment 2: Naive Bayes}\label{assignment-2-naive-bayes}}

Welcome to week two of this specialization. You will learn about Naive
Bayes. Concretely, you will be using Naive Bayes for sentiment analysis
on tweets. Given a tweet, you will decide if it has a positive sentiment
or a negative one. Specifically you will:

\begin{itemize}
\tightlist
\item
  Train a naive bayes model on a sentiment analysis task
\item
  Test using your model
\item
  Compute ratios of positive words to negative words
\item
  Do some error analysis
\item
  Predict on your own tweet
\end{itemize}

You may already be familiar with Naive Bayes and its justification in
terms of conditional probabilities and independence. * In this week's
lectures and assignments we used the ratio of probabilities between
positive and negative sentiment. * This approach gives us simpler
formulas for these 2-way classification tasks.

\hypertarget{important-note-on-submission-to-the-autograder}{%
\subsection{Important Note on Submission to the
AutoGrader}\label{important-note-on-submission-to-the-autograder}}

Before submitting your assignment to the AutoGrader, please make sure
you are not doing the following:

\begin{enumerate}
\def\labelenumi{\arabic{enumi}.}
\tightlist
\item
  You have not added any \emph{extra} \texttt{print} statement(s) in the
  assignment.
\item
  You have not added any \emph{extra} code cell(s) in the assignment.
\item
  You have not changed any of the function parameters.
\item
  You are not using any global variables inside your graded exercises.
  Unless specifically instructed to do so, please refrain from it and
  use the local variables instead.
\item
  You are not changing the assignment code where it is not required,
  like creating \emph{extra} variables.
\end{enumerate}

If you do any of the following, you will get something like,
\texttt{Grader\ not\ found} (or similarly unexpected) error upon
submitting your assignment. Before asking for help/debugging the errors
in your assignment, check for these first. If this is the case, and you
don't remember the changes you have made, you can get a fresh copy of
the assignment by following these
\href{https://www.coursera.org/learn/classification-vector-spaces-in-nlp/supplement/YLuAg/h-ow-to-refresh-your-workspace}{instructions}.

Lets get started!

Load the cell below to import some packages. You may want to browse the
documentation of unfamiliar libraries and functions.

    \begin{tcolorbox}[breakable, size=fbox, boxrule=1pt, pad at break*=1mm,colback=cellbackground, colframe=cellborder]
\prompt{In}{incolor}{1}{\boxspacing}
\begin{Verbatim}[commandchars=\\\{\}]
\PY{k+kn}{from} \PY{n+nn}{utils} \PY{k+kn}{import} \PY{n}{process\PYZus{}tweet}\PY{p}{,} \PY{n}{lookup}
\PY{k+kn}{import} \PY{n+nn}{pdb}
\PY{k+kn}{from} \PY{n+nn}{nltk}\PY{n+nn}{.}\PY{n+nn}{corpus} \PY{k+kn}{import} \PY{n}{stopwords}\PY{p}{,} \PY{n}{twitter\PYZus{}samples}
\PY{k+kn}{import} \PY{n+nn}{numpy} \PY{k}{as} \PY{n+nn}{np}
\PY{k+kn}{import} \PY{n+nn}{pandas} \PY{k}{as} \PY{n+nn}{pd}
\PY{k+kn}{import} \PY{n+nn}{nltk}
\PY{k+kn}{import} \PY{n+nn}{string}
\PY{k+kn}{from} \PY{n+nn}{nltk}\PY{n+nn}{.}\PY{n+nn}{tokenize} \PY{k+kn}{import} \PY{n}{TweetTokenizer}
\PY{k+kn}{from} \PY{n+nn}{os} \PY{k+kn}{import} \PY{n}{getcwd}
\PY{k+kn}{import} \PY{n+nn}{w2\PYZus{}unittest}

\PY{n}{nltk}\PY{o}{.}\PY{n}{download}\PY{p}{(}\PY{l+s+s1}{\PYZsq{}}\PY{l+s+s1}{twitter\PYZus{}samples}\PY{l+s+s1}{\PYZsq{}}\PY{p}{)}
\PY{n}{nltk}\PY{o}{.}\PY{n}{download}\PY{p}{(}\PY{l+s+s1}{\PYZsq{}}\PY{l+s+s1}{stopwords}\PY{l+s+s1}{\PYZsq{}}\PY{p}{)}
\end{Verbatim}
\end{tcolorbox}

    \begin{Verbatim}[commandchars=\\\{\}]
[nltk\_data] Downloading package twitter\_samples to
[nltk\_data]     /home/jovyan/nltk\_data{\ldots}
[nltk\_data]   Unzipping corpora/twitter\_samples.zip.
[nltk\_data] Downloading package stopwords to /home/jovyan/nltk\_data{\ldots}
[nltk\_data]   Unzipping corpora/stopwords.zip.
    \end{Verbatim}

            \begin{tcolorbox}[breakable, size=fbox, boxrule=.5pt, pad at break*=1mm, opacityfill=0]
\prompt{Out}{outcolor}{1}{\boxspacing}
\begin{Verbatim}[commandchars=\\\{\}]
True
\end{Verbatim}
\end{tcolorbox}
        
    If you are running this notebook in your local computer, don't forget to
download the tweeter samples and stopwords from nltk.

\begin{verbatim}
nltk.download('stopwords')
nltk.download('twitter_samples')
\end{verbatim}

    \begin{tcolorbox}[breakable, size=fbox, boxrule=1pt, pad at break*=1mm,colback=cellbackground, colframe=cellborder]
\prompt{In}{incolor}{2}{\boxspacing}
\begin{Verbatim}[commandchars=\\\{\}]
\PY{n}{filePath} \PY{o}{=} \PY{l+s+sa}{f}\PY{l+s+s2}{\PYZdq{}}\PY{l+s+si}{\PYZob{}}\PY{n}{getcwd}\PY{p}{(}\PY{p}{)}\PY{l+s+si}{\PYZcb{}}\PY{l+s+s2}{/../tmp2/}\PY{l+s+s2}{\PYZdq{}}
\PY{n}{nltk}\PY{o}{.}\PY{n}{data}\PY{o}{.}\PY{n}{path}\PY{o}{.}\PY{n}{append}\PY{p}{(}\PY{n}{filePath}\PY{p}{)}
\end{Verbatim}
\end{tcolorbox}

    \begin{tcolorbox}[breakable, size=fbox, boxrule=1pt, pad at break*=1mm,colback=cellbackground, colframe=cellborder]
\prompt{In}{incolor}{3}{\boxspacing}
\begin{Verbatim}[commandchars=\\\{\}]
\PY{c+c1}{\PYZsh{} get the sets of positive and negative tweets}
\PY{n}{all\PYZus{}positive\PYZus{}tweets} \PY{o}{=} \PY{n}{twitter\PYZus{}samples}\PY{o}{.}\PY{n}{strings}\PY{p}{(}\PY{l+s+s1}{\PYZsq{}}\PY{l+s+s1}{positive\PYZus{}tweets.json}\PY{l+s+s1}{\PYZsq{}}\PY{p}{)}
\PY{n}{all\PYZus{}negative\PYZus{}tweets} \PY{o}{=} \PY{n}{twitter\PYZus{}samples}\PY{o}{.}\PY{n}{strings}\PY{p}{(}\PY{l+s+s1}{\PYZsq{}}\PY{l+s+s1}{negative\PYZus{}tweets.json}\PY{l+s+s1}{\PYZsq{}}\PY{p}{)}

\PY{c+c1}{\PYZsh{} split the data into two pieces, one for training and one for testing (validation set)}
\PY{n}{test\PYZus{}pos} \PY{o}{=} \PY{n}{all\PYZus{}positive\PYZus{}tweets}\PY{p}{[}\PY{l+m+mi}{4000}\PY{p}{:}\PY{p}{]}
\PY{n}{train\PYZus{}pos} \PY{o}{=} \PY{n}{all\PYZus{}positive\PYZus{}tweets}\PY{p}{[}\PY{p}{:}\PY{l+m+mi}{4000}\PY{p}{]}
\PY{n}{test\PYZus{}neg} \PY{o}{=} \PY{n}{all\PYZus{}negative\PYZus{}tweets}\PY{p}{[}\PY{l+m+mi}{4000}\PY{p}{:}\PY{p}{]}
\PY{n}{train\PYZus{}neg} \PY{o}{=} \PY{n}{all\PYZus{}negative\PYZus{}tweets}\PY{p}{[}\PY{p}{:}\PY{l+m+mi}{4000}\PY{p}{]}

\PY{n}{train\PYZus{}x} \PY{o}{=} \PY{n}{train\PYZus{}pos} \PY{o}{+} \PY{n}{train\PYZus{}neg}
\PY{n}{test\PYZus{}x} \PY{o}{=} \PY{n}{test\PYZus{}pos} \PY{o}{+} \PY{n}{test\PYZus{}neg}

\PY{c+c1}{\PYZsh{} avoid assumptions about the length of all\PYZus{}positive\PYZus{}tweets}
\PY{n}{train\PYZus{}y} \PY{o}{=} \PY{n}{np}\PY{o}{.}\PY{n}{append}\PY{p}{(}\PY{n}{np}\PY{o}{.}\PY{n}{ones}\PY{p}{(}\PY{n+nb}{len}\PY{p}{(}\PY{n}{train\PYZus{}pos}\PY{p}{)}\PY{p}{)}\PY{p}{,} \PY{n}{np}\PY{o}{.}\PY{n}{zeros}\PY{p}{(}\PY{n+nb}{len}\PY{p}{(}\PY{n}{train\PYZus{}neg}\PY{p}{)}\PY{p}{)}\PY{p}{)}
\PY{n}{test\PYZus{}y} \PY{o}{=} \PY{n}{np}\PY{o}{.}\PY{n}{append}\PY{p}{(}\PY{n}{np}\PY{o}{.}\PY{n}{ones}\PY{p}{(}\PY{n+nb}{len}\PY{p}{(}\PY{n}{test\PYZus{}pos}\PY{p}{)}\PY{p}{)}\PY{p}{,} \PY{n}{np}\PY{o}{.}\PY{n}{zeros}\PY{p}{(}\PY{n+nb}{len}\PY{p}{(}\PY{n}{test\PYZus{}neg}\PY{p}{)}\PY{p}{)}\PY{p}{)}
\end{Verbatim}
\end{tcolorbox}

    \hypertarget{part-1-process-the-data}{%
\section{Part 1: Process the Data}\label{part-1-process-the-data}}

For any machine learning project, once you've gathered the data, the
first step is to process it to make useful inputs to your model. -
\textbf{Remove noise}: You will first want to remove noise from your
data -- that is, remove words that don't tell you much about the
content. These include all common words like `I, you, are, is,
etc\ldots{}' that would not give us enough information on the sentiment.
- We'll also remove stock market tickers, retweet symbols, hyperlinks,
and hashtags because they can not tell you a lot of information on the
sentiment. - You also want to remove all the punctuation from a tweet.
The reason for doing this is because we want to treat words with or
without the punctuation as the same word, instead of treating ``happy'',
``happy?'', ``happy!'', ``happy,'' and ``happy.'' as different words. -
Finally you want to use stemming to only keep track of one variation of
each word. In other words, we'll treat ``motivation'', ``motivated'',
and ``motivate'' similarly by grouping them within the same stem of
``motiv-''.

We have given you the function \texttt{process\_tweet} that does this
for you.

    \begin{tcolorbox}[breakable, size=fbox, boxrule=1pt, pad at break*=1mm,colback=cellbackground, colframe=cellborder]
\prompt{In}{incolor}{4}{\boxspacing}
\begin{Verbatim}[commandchars=\\\{\}]
\PY{n}{custom\PYZus{}tweet} \PY{o}{=} \PY{l+s+s2}{\PYZdq{}}\PY{l+s+s2}{RT @Twitter @chapagain Hello There! Have a great day. :) \PYZsh{}good \PYZsh{}morning http://chapagain.com.np}\PY{l+s+s2}{\PYZdq{}}

\PY{c+c1}{\PYZsh{} print cleaned tweet}
\PY{n+nb}{print}\PY{p}{(}\PY{n}{process\PYZus{}tweet}\PY{p}{(}\PY{n}{custom\PYZus{}tweet}\PY{p}{)}\PY{p}{)}
\end{Verbatim}
\end{tcolorbox}

    \begin{Verbatim}[commandchars=\\\{\}]
['hello', 'great', 'day', ':)', 'good', 'morn']
    \end{Verbatim}

    \hypertarget{part-1.1-implementing-your-helper-functions}{%
\subsection{Part 1.1 Implementing your helper
functions}\label{part-1.1-implementing-your-helper-functions}}

To help you train your naive bayes model, you will need to compute a
dictionary where the keys are a tuple (word, label) and the values are
the corresponding frequency. Note that the labels we'll use here are 1
for positive and 0 for negative.

You will also implement a lookup helper function that takes in the
\texttt{freqs} dictionary, a word, and a label (1 or 0) and returns the
number of times that word and label tuple appears in the collection of
tweets.

For example: given a list of tweets
\texttt{{[}"i\ am\ rather\ excited",\ "you\ are\ rather\ happy"{]}} and
the label 1, the function will return a dictionary that contains the
following key-value pairs:

\{ (``rather'', 1): 2, (``happi'', 1) : 1, (``excit'', 1) : 1 \}

\begin{itemize}
\tightlist
\item
  Notice how for each word in the given string, the same label 1 is
  assigned to each word.
\item
  Notice how the words ``i'' and ``am'' are not saved, since it was
  removed by process\_tweet because it is a stopword.
\item
  Notice how the word ``rather'' appears twice in the list of tweets,
  and so its count value is 2.
\end{itemize}

\hypertarget{instructions}{%
\paragraph{Instructions}\label{instructions}}

Create a function \texttt{count\_tweets} that takes a list of tweets as
input, cleans all of them, and returns a dictionary. - The key in the
dictionary is a tuple containing the stemmed word and its class label,
e.g.~(``happi'',1). - The value the number of times this word appears in
the given collection of tweets (an integer).

    Hints

Please use the \texttt{process\_tweet} function that was imported above,
and then store the words in their respective dictionaries and sets.

You may find it useful to use the \texttt{zip} function to match each
element in \texttt{tweets} with each element in \texttt{ys}.

Remember to check if the key in the dictionary exists before adding that
key to the dictionary, or incrementing its value.

Assume that the \texttt{result} dictionary that is input will contain
clean key-value pairs (you can assume that the values will be integers
that can be incremented). It is good practice to check the datatype
before incrementing the value, but it's not required here.

    \begin{tcolorbox}[breakable, size=fbox, boxrule=1pt, pad at break*=1mm,colback=cellbackground, colframe=cellborder]
\prompt{In}{incolor}{5}{\boxspacing}
\begin{Verbatim}[commandchars=\\\{\}]
\PY{c+c1}{\PYZsh{} UNQ\PYZus{}C1 GRADED FUNCTION: count\PYZus{}tweets}

\PY{k}{def} \PY{n+nf}{count\PYZus{}tweets}\PY{p}{(}\PY{n}{result}\PY{p}{,} \PY{n}{tweets}\PY{p}{,} \PY{n}{ys}\PY{p}{)}\PY{p}{:}
    \PY{l+s+sd}{\PYZsq{}\PYZsq{}\PYZsq{}}
\PY{l+s+sd}{    Input:}
\PY{l+s+sd}{        result: a dictionary that will be used to map each pair to its frequency}
\PY{l+s+sd}{        tweets: a list of tweets}
\PY{l+s+sd}{        ys: a list corresponding to the sentiment of each tweet (either 0 or 1)}
\PY{l+s+sd}{    Output:}
\PY{l+s+sd}{        result: a dictionary mapping each pair to its frequency}
\PY{l+s+sd}{    \PYZsq{}\PYZsq{}\PYZsq{}}
    \PY{c+c1}{\PYZsh{}\PYZsh{}\PYZsh{} START CODE HERE (REPLACE INSTANCES OF \PYZsq{}None\PYZsq{} with your code) \PYZsh{}\PYZsh{}\PYZsh{}}
    \PY{k}{for} \PY{n}{y}\PY{p}{,} \PY{n}{tweet} \PY{o+ow}{in} \PY{n+nb}{zip}\PY{p}{(}\PY{n}{ys}\PY{p}{,} \PY{n}{tweets}\PY{p}{)}\PY{p}{:}
        \PY{k}{for} \PY{n}{word} \PY{o+ow}{in} \PY{n}{process\PYZus{}tweet}\PY{p}{(}\PY{n}{tweet}\PY{p}{)}\PY{p}{:}
            \PY{c+c1}{\PYZsh{} define the key, which is the word and label tuple}
            \PY{n}{pair} \PY{o}{=} \PY{p}{(}\PY{n}{word}\PY{p}{,} \PY{n}{y}\PY{p}{)}

            \PY{c+c1}{\PYZsh{} if the key exists in the dictionary, increment the count}
            \PY{k}{if} \PY{n}{pair} \PY{o+ow}{in} \PY{n}{result}\PY{p}{:}
                \PY{n}{result}\PY{p}{[}\PY{n}{pair}\PY{p}{]} \PY{o}{+}\PY{o}{=} \PY{l+m+mi}{1}

            \PY{c+c1}{\PYZsh{} else, if the key is new, add it to the dictionary and set the count to 1}
            \PY{k}{else}\PY{p}{:}
                \PY{n}{result}\PY{p}{[}\PY{n}{pair}\PY{p}{]} \PY{o}{=} \PY{l+m+mi}{1}
    \PY{c+c1}{\PYZsh{}\PYZsh{}\PYZsh{} END CODE HERE \PYZsh{}\PYZsh{}\PYZsh{}}

    \PY{k}{return} \PY{n}{result}
\end{Verbatim}
\end{tcolorbox}

    \begin{tcolorbox}[breakable, size=fbox, boxrule=1pt, pad at break*=1mm,colback=cellbackground, colframe=cellborder]
\prompt{In}{incolor}{6}{\boxspacing}
\begin{Verbatim}[commandchars=\\\{\}]
\PY{c+c1}{\PYZsh{} Testing your function}

\PY{n}{result} \PY{o}{=} \PY{p}{\PYZob{}}\PY{p}{\PYZcb{}}
\PY{n}{tweets} \PY{o}{=} \PY{p}{[}\PY{l+s+s1}{\PYZsq{}}\PY{l+s+s1}{i am happy}\PY{l+s+s1}{\PYZsq{}}\PY{p}{,} \PY{l+s+s1}{\PYZsq{}}\PY{l+s+s1}{i am tricked}\PY{l+s+s1}{\PYZsq{}}\PY{p}{,} \PY{l+s+s1}{\PYZsq{}}\PY{l+s+s1}{i am sad}\PY{l+s+s1}{\PYZsq{}}\PY{p}{,} \PY{l+s+s1}{\PYZsq{}}\PY{l+s+s1}{i am tired}\PY{l+s+s1}{\PYZsq{}}\PY{p}{,} \PY{l+s+s1}{\PYZsq{}}\PY{l+s+s1}{i am tired}\PY{l+s+s1}{\PYZsq{}}\PY{p}{]}
\PY{n}{ys} \PY{o}{=} \PY{p}{[}\PY{l+m+mi}{1}\PY{p}{,} \PY{l+m+mi}{0}\PY{p}{,} \PY{l+m+mi}{0}\PY{p}{,} \PY{l+m+mi}{0}\PY{p}{,} \PY{l+m+mi}{0}\PY{p}{]}
\PY{n}{count\PYZus{}tweets}\PY{p}{(}\PY{n}{result}\PY{p}{,} \PY{n}{tweets}\PY{p}{,} \PY{n}{ys}\PY{p}{)}
\end{Verbatim}
\end{tcolorbox}

            \begin{tcolorbox}[breakable, size=fbox, boxrule=.5pt, pad at break*=1mm, opacityfill=0]
\prompt{Out}{outcolor}{6}{\boxspacing}
\begin{Verbatim}[commandchars=\\\{\}]
\{('happi', 1): 1, ('trick', 0): 1, ('sad', 0): 1, ('tire', 0): 2\}
\end{Verbatim}
\end{tcolorbox}
        
    \textbf{Expected Output}: \{(`happi', 1): 1, (`trick', 0): 1, (`sad',
0): 1, (`tire', 0): 2\}

    \begin{tcolorbox}[breakable, size=fbox, boxrule=1pt, pad at break*=1mm,colback=cellbackground, colframe=cellborder]
\prompt{In}{incolor}{7}{\boxspacing}
\begin{Verbatim}[commandchars=\\\{\}]
\PY{c+c1}{\PYZsh{} Test your function}
\PY{n}{w2\PYZus{}unittest}\PY{o}{.}\PY{n}{test\PYZus{}count\PYZus{}tweets}\PY{p}{(}\PY{n}{count\PYZus{}tweets}\PY{p}{)}
\end{Verbatim}
\end{tcolorbox}

    \begin{Verbatim}[commandchars=\\\{\}]
\textcolor{ansi-green-intense}{ All tests passed}
    \end{Verbatim}

    \hypertarget{part-2-train-your-model-using-naive-bayes}{%
\section{Part 2: Train your model using Naive
Bayes}\label{part-2-train-your-model-using-naive-bayes}}

Naive bayes is an algorithm that could be used for sentiment analysis.
It takes a short time to train and also has a short prediction time.

\hypertarget{so-how-do-you-train-a-naive-bayes-classifier}{%
\paragraph{So how do you train a Naive Bayes
classifier?}\label{so-how-do-you-train-a-naive-bayes-classifier}}

\begin{itemize}
\tightlist
\item
  The first part of training a naive bayes classifier is to identify the
  number of classes that you have.
\item
  You will create a probability for each class. \(P(D_{pos})\) is the
  probability that the document is positive. \(P(D_{neg})\) is the
  probability that the document is negative. Use the formulas as follows
  and store the values in a dictionary:
\end{itemize}

\[P(D_{pos}) = \frac{D_{pos}}{D}\tag{1}\]

\[P(D_{neg}) = \frac{D_{neg}}{D}\tag{2}\]

Where \(D\) is the total number of documents, or tweets in this case,
\(D_{pos}\) is the total number of positive tweets and \(D_{neg}\) is
the total number of negative tweets.

    \hypertarget{prior-and-logprior}{%
\paragraph{Prior and Logprior}\label{prior-and-logprior}}

The prior probability represents the underlying probability in the
target population that a tweet is positive versus negative. In other
words, if we had no specific information and blindly picked a tweet out
of the population set, what is the probability that it will be positive
versus that it will be negative? That is the ``prior''.

The prior is the ratio of the probabilities
\(\frac{P(D_{pos})}{P(D_{neg})}\). We can take the log of the prior to
rescale it, and we'll call this the logprior

\[\text{logprior} = log \left( \frac{P(D_{pos})}{P(D_{neg})} \right) = log \left( \frac{D_{pos}}{D_{neg}} \right)\].

Note that \(log(\frac{A}{B})\) is the same as \(log(A) - log(B)\). So
the logprior can also be calculated as the difference between two logs:

\[\text{logprior} = \log (P(D_{pos})) - \log (P(D_{neg})) = \log (D_{pos}) - \log (D_{neg})\tag{3}\]

    \hypertarget{positive-and-negative-probability-of-a-word}{%
\paragraph{Positive and Negative Probability of a
Word}\label{positive-and-negative-probability-of-a-word}}

To compute the positive probability and the negative probability for a
specific word in the vocabulary, we'll use the following inputs:

\begin{itemize}
\tightlist
\item
  \(freq_{pos}\) and \(freq_{neg}\) are the frequencies of that specific
  word in the positive or negative class. In other words, the positive
  frequency of a word is the number of times the word is counted with
  the label of 1.
\item
  \(N_{pos}\) and \(N_{neg}\) are the total number of positive and
  negative words for all documents (for all tweets), respectively.
\item
  \(V\) is the number of unique words in the entire set of documents,
  for all classes, whether positive or negative.
\end{itemize}

We'll use these to compute the positive and negative probability for a
specific word using this formula:

\[ P(W_{pos}) = \frac{freq_{pos} + 1}{N_{pos} + V}\tag{4} \]
\[ P(W_{neg}) = \frac{freq_{neg} + 1}{N_{neg} + V}\tag{5} \]

Notice that we add the ``+1'' in the numerator for additive smoothing.
This \href{https://en.wikipedia.org/wiki/Additive_smoothing}{wiki
article} explains more about additive smoothing.

    \hypertarget{log-likelihood}{%
\paragraph{Log likelihood}\label{log-likelihood}}

To compute the loglikelihood of that very same word, we can implement
the following equations:

\[\text{loglikelihood} = \log \left(\frac{P(W_{pos})}{P(W_{neg})} \right)\tag{6}\]

    \hypertarget{create-freqs-dictionary}{%
\subparagraph{\texorpdfstring{Create \texttt{freqs}
dictionary}{Create freqs dictionary}}\label{create-freqs-dictionary}}

\begin{itemize}
\tightlist
\item
  Given your \texttt{count\_tweets} function, you can compute a
  dictionary called \texttt{freqs} that contains all the frequencies.
\item
  In this \texttt{freqs} dictionary, the key is the tuple (word, label)
\item
  The value is the number of times it has appeared.
\end{itemize}

We will use this dictionary in several parts of this assignment.

    \begin{tcolorbox}[breakable, size=fbox, boxrule=1pt, pad at break*=1mm,colback=cellbackground, colframe=cellborder]
\prompt{In}{incolor}{8}{\boxspacing}
\begin{Verbatim}[commandchars=\\\{\}]
\PY{c+c1}{\PYZsh{} Build the freqs dictionary for later uses}
\PY{n}{freqs} \PY{o}{=} \PY{n}{count\PYZus{}tweets}\PY{p}{(}\PY{p}{\PYZob{}}\PY{p}{\PYZcb{}}\PY{p}{,} \PY{n}{train\PYZus{}x}\PY{p}{,} \PY{n}{train\PYZus{}y}\PY{p}{)}
\end{Verbatim}
\end{tcolorbox}

    \hypertarget{instructions}{%
\paragraph{Instructions}\label{instructions}}

Given a freqs dictionary, \texttt{train\_x} (a list of tweets) and a
\texttt{train\_y} (a list of labels for each tweet), implement a naive
bayes classifier.

\hypertarget{calculate-v}{%
\subparagraph{\texorpdfstring{Calculate
\(V\)}{Calculate V}}\label{calculate-v}}

\begin{itemize}
\tightlist
\item
  You can then compute the number of unique words that appear in the
  \texttt{freqs} dictionary to get your \(V\) (you can use the
  \texttt{set} function).
\end{itemize}

\hypertarget{calculate-freq_pos-and-freq_neg}{%
\subparagraph{\texorpdfstring{Calculate \(freq_{pos}\) and
\(freq_{neg}\)}{Calculate freq\_\{pos\} and freq\_\{neg\}}}\label{calculate-freq_pos-and-freq_neg}}

\begin{itemize}
\tightlist
\item
  Using your \texttt{freqs} dictionary, you can compute the positive and
  negative frequency of each word \(freq_{pos}\) and \(freq_{neg}\).
\end{itemize}

\hypertarget{calculate-n_pos-and-n_neg}{%
\subparagraph{\texorpdfstring{Calculate \(N_{pos}\), and
\(N_{neg}\)}{Calculate N\_\{pos\}, and N\_\{neg\}}}\label{calculate-n_pos-and-n_neg}}

\begin{itemize}
\tightlist
\item
  Using \texttt{freqs} dictionary, you can also compute the total number
  of positive words and total number of negative words \(N_{pos}\) and
  \(N_{neg}\).
\end{itemize}

\hypertarget{calculate-d-d_pos-d_neg}{%
\subparagraph{\texorpdfstring{Calculate \(D\), \(D_{pos}\),
\(D_{neg}\)}{Calculate D, D\_\{pos\}, D\_\{neg\}}}\label{calculate-d-d_pos-d_neg}}

\begin{itemize}
\tightlist
\item
  Using the \texttt{train\_y} input list of labels, calculate the number
  of documents (tweets) \(D\), as well as the number of positive
  documents (tweets) \(D_{pos}\) and number of negative documents
  (tweets) \(D_{neg}\).
\item
  Calculate the probability that a document (tweet) is positive
  \(P(D_{pos})\), and the probability that a document (tweet) is
  negative \(P(D_{neg})\)
\end{itemize}

\hypertarget{calculate-the-logprior}{%
\subparagraph{Calculate the logprior}\label{calculate-the-logprior}}

\begin{itemize}
\tightlist
\item
  the logprior is \(log(D_{pos}) - log(D_{neg})\)
\end{itemize}

\hypertarget{calculate-log-likelihood}{%
\subparagraph{Calculate log likelihood}\label{calculate-log-likelihood}}

\begin{itemize}
\tightlist
\item
  Finally, you can iterate over each word in the vocabulary, use your
  \texttt{lookup} function to get the positive frequencies,
  \(freq_{pos}\), and the negative frequencies, \(freq_{neg}\), for that
  specific word.
\item
  Compute the positive probability of each word \(P(W_{pos})\), negative
  probability of each word \(P(W_{neg})\) using equations 4 \& 5.
\end{itemize}

\[ P(W_{pos}) = \frac{freq_{pos} + 1}{N_{pos} + V}\tag{4} \]
\[ P(W_{neg}) = \frac{freq_{neg} + 1}{N_{neg} + V}\tag{5} \]

\textbf{Note:} We'll use a dictionary to store the log likelihoods for
each word. The key is the word, the value is the log likelihood of that
word).

\begin{itemize}
\tightlist
\item
  You can then compute the loglikelihood:
  \(log \left( \frac{P(W_{pos})}{P(W_{neg})} \right)\).
\end{itemize}

    \begin{tcolorbox}[breakable, size=fbox, boxrule=1pt, pad at break*=1mm,colback=cellbackground, colframe=cellborder]
\prompt{In}{incolor}{9}{\boxspacing}
\begin{Verbatim}[commandchars=\\\{\}]
\PY{c+c1}{\PYZsh{} UNQ\PYZus{}C2 GRADED FUNCTION: train\PYZus{}naive\PYZus{}bayes}

\PY{k}{def} \PY{n+nf}{train\PYZus{}naive\PYZus{}bayes}\PY{p}{(}\PY{n}{freqs}\PY{p}{,} \PY{n}{train\PYZus{}x}\PY{p}{,} \PY{n}{train\PYZus{}y}\PY{p}{)}\PY{p}{:}
    \PY{l+s+sd}{\PYZsq{}\PYZsq{}\PYZsq{}}
\PY{l+s+sd}{    Input:}
\PY{l+s+sd}{        freqs: dictionary from (word, label) to how often the word appears}
\PY{l+s+sd}{        train\PYZus{}x: a list of tweets}
\PY{l+s+sd}{        train\PYZus{}y: a list of labels correponding to the tweets (0,1)}
\PY{l+s+sd}{    Output:}
\PY{l+s+sd}{        logprior: the log prior. (equation 3 above)}
\PY{l+s+sd}{        loglikelihood: the log likelihood of you Naive bayes equation. (equation 6 above)}
\PY{l+s+sd}{    \PYZsq{}\PYZsq{}\PYZsq{}}
    \PY{n}{loglikelihood} \PY{o}{=} \PY{p}{\PYZob{}}\PY{p}{\PYZcb{}}
    \PY{n}{logprior} \PY{o}{=} \PY{l+m+mi}{0}

    \PY{c+c1}{\PYZsh{}\PYZsh{}\PYZsh{} START CODE HERE (REPLACE INSTANCES OF \PYZsq{}None\PYZsq{} with your code) \PYZsh{}\PYZsh{}\PYZsh{}}

    \PY{c+c1}{\PYZsh{} calculate V, the number of unique words in the vocabulary}
    \PY{n}{vocab} \PY{o}{=} \PY{n+nb}{set}\PY{p}{(}\PY{p}{[}\PY{n}{pair}\PY{p}{[}\PY{l+m+mi}{0}\PY{p}{]} \PY{k}{for} \PY{n}{pair} \PY{o+ow}{in} \PY{n}{freqs}\PY{o}{.}\PY{n}{keys}\PY{p}{(}\PY{p}{)}\PY{p}{]}\PY{p}{)}
    \PY{n}{V} \PY{o}{=} \PY{n+nb}{len}\PY{p}{(}\PY{n}{vocab}\PY{p}{)}

    \PY{c+c1}{\PYZsh{} calculate N\PYZus{}pos and N\PYZus{}neg}
    \PY{n}{N\PYZus{}pos} \PY{o}{=} \PY{n}{N\PYZus{}neg} \PY{o}{=} \PY{l+m+mi}{0}
    \PY{k}{for} \PY{n}{pair} \PY{o+ow}{in} \PY{n}{freqs}\PY{o}{.}\PY{n}{keys}\PY{p}{(}\PY{p}{)}\PY{p}{:}
        \PY{c+c1}{\PYZsh{} if the label is positive (greater than zero)}
        \PY{k}{if} \PY{n}{pair}\PY{p}{[}\PY{l+m+mi}{1}\PY{p}{]} \PY{o}{\PYZgt{}} \PY{l+m+mi}{0}\PY{p}{:}

            \PY{c+c1}{\PYZsh{} Increment the number of positive words by the count for this (word, label) pair}
            \PY{n}{N\PYZus{}pos} \PY{o}{+}\PY{o}{=} \PY{n}{freqs}\PY{p}{[}\PY{n}{pair}\PY{p}{]}

        \PY{c+c1}{\PYZsh{} else, the label is negative}
        \PY{k}{else}\PY{p}{:}

            \PY{c+c1}{\PYZsh{} increment the number of negative words by the count for this (word,label) pair}
            \PY{n}{N\PYZus{}neg} \PY{o}{+}\PY{o}{=} \PY{n}{freqs}\PY{p}{[}\PY{n}{pair}\PY{p}{]}

    \PY{c+c1}{\PYZsh{} Calculate D, the number of documents}
    \PY{n}{D} \PY{o}{=} \PY{n+nb}{len}\PY{p}{(}\PY{n}{train\PYZus{}y}\PY{p}{)}

    \PY{c+c1}{\PYZsh{} Calculate D\PYZus{}pos, the number of positive documents (*hint: use sum(\PYZlt{}np\PYZus{}array\PYZgt{}))}
    \PY{n}{D\PYZus{}pos} \PY{o}{=} \PY{n+nb}{len}\PY{p}{(}\PY{n+nb}{list}\PY{p}{(}\PY{n+nb}{filter}\PY{p}{(}\PY{k}{lambda} \PY{n}{x}\PY{p}{:} \PY{n}{x} \PY{o}{\PYZgt{}} \PY{l+m+mi}{0}\PY{p}{,} \PY{n}{train\PYZus{}y}\PY{p}{)}\PY{p}{)}\PY{p}{)}

    \PY{c+c1}{\PYZsh{} Calculate D\PYZus{}neg, the number of negative documents (*hint: compute using D and D\PYZus{}pos)}
    \PY{n}{D\PYZus{}neg} \PY{o}{=} \PY{n+nb}{len}\PY{p}{(}\PY{n+nb}{list}\PY{p}{(}\PY{n+nb}{filter}\PY{p}{(}\PY{k}{lambda} \PY{n}{x}\PY{p}{:} \PY{n}{x} \PY{o}{\PYZlt{}}\PY{o}{=} \PY{l+m+mi}{0}\PY{p}{,} \PY{n}{train\PYZus{}y}\PY{p}{)}\PY{p}{)}\PY{p}{)}

    \PY{c+c1}{\PYZsh{} Calculate logprior}
    \PY{n}{logprior} \PY{o}{=} \PY{n}{np}\PY{o}{.}\PY{n}{log}\PY{p}{(}\PY{n}{D\PYZus{}pos}\PY{p}{)} \PY{o}{\PYZhy{}} \PY{n}{np}\PY{o}{.}\PY{n}{log}\PY{p}{(}\PY{n}{D\PYZus{}neg}\PY{p}{)}

    \PY{c+c1}{\PYZsh{} For each word in the vocabulary...}
    \PY{k}{for} \PY{n}{word} \PY{o+ow}{in} \PY{n}{vocab}\PY{p}{:}
        \PY{c+c1}{\PYZsh{} get the positive and negative frequency of the word}
        \PY{n}{freq\PYZus{}pos} \PY{o}{=} \PY{n}{lookup}\PY{p}{(}\PY{n}{freqs}\PY{p}{,} \PY{n}{word}\PY{p}{,} \PY{l+m+mi}{1}\PY{p}{)}
        \PY{n}{freq\PYZus{}neg} \PY{o}{=} \PY{n}{lookup}\PY{p}{(}\PY{n}{freqs}\PY{p}{,} \PY{n}{word}\PY{p}{,} \PY{l+m+mi}{0}\PY{p}{)}

        \PY{c+c1}{\PYZsh{} calculate the probability that each word is positive, and negative}
        \PY{n}{p\PYZus{}w\PYZus{}pos} \PY{o}{=} \PY{p}{(}\PY{n}{freq\PYZus{}pos} \PY{o}{+} \PY{l+m+mi}{1}\PY{p}{)} \PY{o}{/} \PY{p}{(}\PY{n}{N\PYZus{}pos} \PY{o}{+} \PY{n}{V}\PY{p}{)}
        \PY{n}{p\PYZus{}w\PYZus{}neg} \PY{o}{=} \PY{p}{(}\PY{n}{freq\PYZus{}neg} \PY{o}{+} \PY{l+m+mi}{1}\PY{p}{)} \PY{o}{/} \PY{p}{(}\PY{n}{N\PYZus{}neg} \PY{o}{+} \PY{n}{V}\PY{p}{)}

        \PY{c+c1}{\PYZsh{} calculate the log likelihood of the word}
        \PY{n}{loglikelihood}\PY{p}{[}\PY{n}{word}\PY{p}{]} \PY{o}{=} \PY{n}{np}\PY{o}{.}\PY{n}{log}\PY{p}{(}\PY{n}{p\PYZus{}w\PYZus{}pos} \PY{o}{/} \PY{n}{p\PYZus{}w\PYZus{}neg}\PY{p}{)}

    \PY{c+c1}{\PYZsh{}\PYZsh{}\PYZsh{} END CODE HERE \PYZsh{}\PYZsh{}\PYZsh{}}

    \PY{k}{return} \PY{n}{logprior}\PY{p}{,} \PY{n}{loglikelihood}
\end{Verbatim}
\end{tcolorbox}

    \begin{tcolorbox}[breakable, size=fbox, boxrule=1pt, pad at break*=1mm,colback=cellbackground, colframe=cellborder]
\prompt{In}{incolor}{10}{\boxspacing}
\begin{Verbatim}[commandchars=\\\{\}]
\PY{c+c1}{\PYZsh{} UNQ\PYZus{}C3 (UNIQUE CELL IDENTIFIER, DO NOT EDIT)}
\PY{n}{logprior}\PY{p}{,} \PY{n}{loglikelihood} \PY{o}{=} \PY{n}{train\PYZus{}naive\PYZus{}bayes}\PY{p}{(}\PY{n}{freqs}\PY{p}{,} \PY{n}{train\PYZus{}x}\PY{p}{,} \PY{n}{train\PYZus{}y}\PY{p}{)}
\PY{n+nb}{print}\PY{p}{(}\PY{n}{logprior}\PY{p}{)}
\PY{n+nb}{print}\PY{p}{(}\PY{n+nb}{len}\PY{p}{(}\PY{n}{loglikelihood}\PY{p}{)}\PY{p}{)}
\end{Verbatim}
\end{tcolorbox}

    \begin{Verbatim}[commandchars=\\\{\}]
0.0
9165
    \end{Verbatim}

    \textbf{Expected Output}:

0.0

9165

    \begin{tcolorbox}[breakable, size=fbox, boxrule=1pt, pad at break*=1mm,colback=cellbackground, colframe=cellborder]
\prompt{In}{incolor}{11}{\boxspacing}
\begin{Verbatim}[commandchars=\\\{\}]
\PY{c+c1}{\PYZsh{} Test your function}
\PY{n}{w2\PYZus{}unittest}\PY{o}{.}\PY{n}{test\PYZus{}train\PYZus{}naive\PYZus{}bayes}\PY{p}{(}\PY{n}{train\PYZus{}naive\PYZus{}bayes}\PY{p}{,} \PY{n}{freqs}\PY{p}{,} \PY{n}{train\PYZus{}x}\PY{p}{,} \PY{n}{train\PYZus{}y}\PY{p}{)}
\end{Verbatim}
\end{tcolorbox}

    \begin{Verbatim}[commandchars=\\\{\}]
\textcolor{ansi-green-intense}{ All tests passed}
    \end{Verbatim}

    \hypertarget{part-3-test-your-naive-bayes}{%
\section{Part 3: Test your naive
bayes}\label{part-3-test-your-naive-bayes}}

Now that we have the \texttt{logprior} and \texttt{loglikelihood}, we
can test the naive bayes function by making predicting on some tweets!

\hypertarget{implement-naive_bayes_predict}{%
\paragraph{\texorpdfstring{Implement
\texttt{naive\_bayes\_predict}}{Implement naive\_bayes\_predict}}\label{implement-naive_bayes_predict}}

\textbf{Instructions}: Implement the \texttt{naive\_bayes\_predict}
function to make predictions on tweets. * The function takes in the
\texttt{tweet}, \texttt{logprior}, \texttt{loglikelihood}. * It returns
the probability that the tweet belongs to the positive or negative
class. * For each tweet, sum up loglikelihoods of each word in the
tweet. * Also add the logprior to this sum to get the predicted
sentiment of that tweet.

\[ p = logprior + \sum_i^N (loglikelihood_i)\]

\hypertarget{note}{%
\paragraph{Note}\label{note}}

Note we calculate the prior from the training data, and that the
training data is evenly split between positive and negative labels (4000
positive and 4000 negative tweets). This means that the ratio of
positive to negative 1, and the logprior is 0.

The value of 0.0 means that when we add the logprior to the log
likelihood, we're just adding zero to the log likelihood. However,
please remember to include the logprior, because whenever the data is
not perfectly balanced, the logprior will be a non-zero value.

    \begin{tcolorbox}[breakable, size=fbox, boxrule=1pt, pad at break*=1mm,colback=cellbackground, colframe=cellborder]
\prompt{In}{incolor}{12}{\boxspacing}
\begin{Verbatim}[commandchars=\\\{\}]
\PY{c+c1}{\PYZsh{} UNQ\PYZus{}C4 GRADED FUNCTION: naive\PYZus{}bayes\PYZus{}predict}

\PY{k}{def} \PY{n+nf}{naive\PYZus{}bayes\PYZus{}predict}\PY{p}{(}\PY{n}{tweet}\PY{p}{,} \PY{n}{logprior}\PY{p}{,} \PY{n}{loglikelihood}\PY{p}{)}\PY{p}{:}
    \PY{l+s+sd}{\PYZsq{}\PYZsq{}\PYZsq{}}
\PY{l+s+sd}{    Input:}
\PY{l+s+sd}{        tweet: a string}
\PY{l+s+sd}{        logprior: a number}
\PY{l+s+sd}{        loglikelihood: a dictionary of words mapping to numbers}
\PY{l+s+sd}{    Output:}
\PY{l+s+sd}{        p: the sum of all the logliklihoods of each word in the tweet (if found in the dictionary) + logprior (a number)}

\PY{l+s+sd}{    \PYZsq{}\PYZsq{}\PYZsq{}}
    \PY{c+c1}{\PYZsh{}\PYZsh{}\PYZsh{} START CODE HERE (REPLACE INSTANCES OF \PYZsq{}None\PYZsq{} with your code) \PYZsh{}\PYZsh{}\PYZsh{}}
    \PY{c+c1}{\PYZsh{} process the tweet to get a list of words}
    \PY{n}{word\PYZus{}l} \PY{o}{=} \PY{n}{process\PYZus{}tweet}\PY{p}{(}\PY{n}{tweet}\PY{p}{)}

    \PY{c+c1}{\PYZsh{} initialize probability to zero}
    \PY{n}{p} \PY{o}{=} \PY{l+m+mi}{0}

    \PY{c+c1}{\PYZsh{} add the logprior}
    \PY{n}{p} \PY{o}{+}\PY{o}{=} \PY{n}{logprior}

    \PY{k}{for} \PY{n}{word} \PY{o+ow}{in} \PY{n}{word\PYZus{}l}\PY{p}{:}

        \PY{c+c1}{\PYZsh{} check if the word exists in the loglikelihood dictionary}
        \PY{k}{if} \PY{n}{word} \PY{o+ow}{in} \PY{n}{loglikelihood}\PY{p}{:}
            \PY{c+c1}{\PYZsh{} add the log likelihood of that word to the probability}
            \PY{n}{p} \PY{o}{+}\PY{o}{=} \PY{n}{loglikelihood}\PY{p}{[}\PY{n}{word}\PY{p}{]}

    \PY{c+c1}{\PYZsh{}\PYZsh{}\PYZsh{} END CODE HERE \PYZsh{}\PYZsh{}\PYZsh{}}

    \PY{k}{return} \PY{n}{p}
\end{Verbatim}
\end{tcolorbox}

    \begin{tcolorbox}[breakable, size=fbox, boxrule=1pt, pad at break*=1mm,colback=cellbackground, colframe=cellborder]
\prompt{In}{incolor}{13}{\boxspacing}
\begin{Verbatim}[commandchars=\\\{\}]
\PY{c+c1}{\PYZsh{} UNQ\PYZus{}C5 (UNIQUE CELL IDENTIFIER, DO NOT EDIT)}
\PY{c+c1}{\PYZsh{} Experiment with your own tweet.}
\PY{n}{my\PYZus{}tweet} \PY{o}{=} \PY{l+s+s1}{\PYZsq{}}\PY{l+s+s1}{She smiled.}\PY{l+s+s1}{\PYZsq{}}
\PY{n}{p} \PY{o}{=} \PY{n}{naive\PYZus{}bayes\PYZus{}predict}\PY{p}{(}\PY{n}{my\PYZus{}tweet}\PY{p}{,} \PY{n}{logprior}\PY{p}{,} \PY{n}{loglikelihood}\PY{p}{)}
\PY{n+nb}{print}\PY{p}{(}\PY{l+s+s1}{\PYZsq{}}\PY{l+s+s1}{The expected output is}\PY{l+s+s1}{\PYZsq{}}\PY{p}{,} \PY{n}{p}\PY{p}{)}
\end{Verbatim}
\end{tcolorbox}

    \begin{Verbatim}[commandchars=\\\{\}]
The expected output is 1.5577981920239676
    \end{Verbatim}

    \textbf{Expected Output}: - The expected output is around 1.55 - The
sentiment is positive.

    \begin{tcolorbox}[breakable, size=fbox, boxrule=1pt, pad at break*=1mm,colback=cellbackground, colframe=cellborder]
\prompt{In}{incolor}{14}{\boxspacing}
\begin{Verbatim}[commandchars=\\\{\}]
\PY{c+c1}{\PYZsh{} Test your function}
\PY{n}{w2\PYZus{}unittest}\PY{o}{.}\PY{n}{test\PYZus{}naive\PYZus{}bayes\PYZus{}predict}\PY{p}{(}\PY{n}{naive\PYZus{}bayes\PYZus{}predict}\PY{p}{)}
\end{Verbatim}
\end{tcolorbox}

    \begin{Verbatim}[commandchars=\\\{\}]
\textcolor{ansi-green-intense}{ All tests passed}
    \end{Verbatim}

    \hypertarget{implement-test_naive_bayes}{%
\paragraph{Implement
test\_naive\_bayes}\label{implement-test_naive_bayes}}

\textbf{Instructions}: * Implement \texttt{test\_naive\_bayes} to check
the accuracy of your predictions. * The function takes in your
\texttt{test\_x}, \texttt{test\_y}, log\_prior, and loglikelihood * It
returns the accuracy of your model. * First, use
\texttt{naive\_bayes\_predict} function to make predictions for each
tweet in text\_x.

    \begin{tcolorbox}[breakable, size=fbox, boxrule=1pt, pad at break*=1mm,colback=cellbackground, colframe=cellborder]
\prompt{In}{incolor}{15}{\boxspacing}
\begin{Verbatim}[commandchars=\\\{\}]
\PY{c+c1}{\PYZsh{} UNQ\PYZus{}C6 GRADED FUNCTION: test\PYZus{}naive\PYZus{}bayes}

\PY{k}{def} \PY{n+nf}{test\PYZus{}naive\PYZus{}bayes}\PY{p}{(}\PY{n}{test\PYZus{}x}\PY{p}{,} \PY{n}{test\PYZus{}y}\PY{p}{,} \PY{n}{logprior}\PY{p}{,} \PY{n}{loglikelihood}\PY{p}{,} \PY{n}{naive\PYZus{}bayes\PYZus{}predict}\PY{o}{=}\PY{n}{naive\PYZus{}bayes\PYZus{}predict}\PY{p}{)}\PY{p}{:}
    \PY{l+s+sd}{\PYZdq{}\PYZdq{}\PYZdq{}}
\PY{l+s+sd}{    Input:}
\PY{l+s+sd}{        test\PYZus{}x: A list of tweets}
\PY{l+s+sd}{        test\PYZus{}y: the corresponding labels for the list of tweets}
\PY{l+s+sd}{        logprior: the logprior}
\PY{l+s+sd}{        loglikelihood: a dictionary with the loglikelihoods for each word}
\PY{l+s+sd}{    Output:}
\PY{l+s+sd}{        accuracy: (\PYZsh{} of tweets classified correctly)/(total \PYZsh{} of tweets)}
\PY{l+s+sd}{    \PYZdq{}\PYZdq{}\PYZdq{}}
    \PY{n}{accuracy} \PY{o}{=} \PY{l+m+mi}{0}  \PY{c+c1}{\PYZsh{} return this properly}

    \PY{c+c1}{\PYZsh{}\PYZsh{}\PYZsh{} START CODE HERE (REPLACE INSTANCES OF \PYZsq{}None\PYZsq{} with your code) \PYZsh{}\PYZsh{}\PYZsh{}}
    \PY{n}{y\PYZus{}hats} \PY{o}{=} \PY{p}{[}\PY{p}{]}
    \PY{k}{for} \PY{n}{tweet} \PY{o+ow}{in} \PY{n}{test\PYZus{}x}\PY{p}{:}
        \PY{c+c1}{\PYZsh{} if the prediction is \PYZgt{} 0}
        \PY{k}{if} \PY{n}{naive\PYZus{}bayes\PYZus{}predict}\PY{p}{(}\PY{n}{tweet}\PY{p}{,} \PY{n}{logprior}\PY{p}{,} \PY{n}{loglikelihood}\PY{p}{)} \PY{o}{\PYZgt{}} \PY{l+m+mi}{0}\PY{p}{:}
            \PY{c+c1}{\PYZsh{} the predicted class is 1}
            \PY{n}{y\PYZus{}hat\PYZus{}i} \PY{o}{=} \PY{l+m+mi}{1}
        \PY{k}{else}\PY{p}{:}
            \PY{c+c1}{\PYZsh{} otherwise the predicted class is 0}
            \PY{n}{y\PYZus{}hat\PYZus{}i} \PY{o}{=} \PY{l+m+mi}{0}

        \PY{c+c1}{\PYZsh{} append the predicted class to the list y\PYZus{}hats}
        \PY{n}{y\PYZus{}hats}\PY{o}{.}\PY{n}{append}\PY{p}{(}\PY{n}{y\PYZus{}hat\PYZus{}i}\PY{p}{)}

    \PY{c+c1}{\PYZsh{} error is the average of the absolute values of the differences between y\PYZus{}hats and test\PYZus{}y}
    \PY{n}{error} \PY{o}{=} \PY{n}{np}\PY{o}{.}\PY{n}{mean}\PY{p}{(}\PY{n}{np}\PY{o}{.}\PY{n}{absolute}\PY{p}{(}\PY{n}{y\PYZus{}hats} \PY{o}{\PYZhy{}} \PY{n}{test\PYZus{}y}\PY{p}{)}\PY{p}{)}

    \PY{c+c1}{\PYZsh{} Accuracy is 1 minus the error}
    \PY{n}{accuracy} \PY{o}{=} \PY{l+m+mi}{1} \PY{o}{\PYZhy{}} \PY{n}{error}

    \PY{c+c1}{\PYZsh{}\PYZsh{}\PYZsh{} END CODE HERE \PYZsh{}\PYZsh{}\PYZsh{}}

    \PY{k}{return} \PY{n}{accuracy}
\end{Verbatim}
\end{tcolorbox}

    \begin{tcolorbox}[breakable, size=fbox, boxrule=1pt, pad at break*=1mm,colback=cellbackground, colframe=cellborder]
\prompt{In}{incolor}{16}{\boxspacing}
\begin{Verbatim}[commandchars=\\\{\}]
\PY{n+nb}{print}\PY{p}{(}\PY{l+s+s2}{\PYZdq{}}\PY{l+s+s2}{Naive Bayes accuracy = }\PY{l+s+si}{\PYZpc{}0.4f}\PY{l+s+s2}{\PYZdq{}} \PY{o}{\PYZpc{}}
      \PY{p}{(}\PY{n}{test\PYZus{}naive\PYZus{}bayes}\PY{p}{(}\PY{n}{test\PYZus{}x}\PY{p}{,} \PY{n}{test\PYZus{}y}\PY{p}{,} \PY{n}{logprior}\PY{p}{,} \PY{n}{loglikelihood}\PY{p}{)}\PY{p}{)}\PY{p}{)}
\end{Verbatim}
\end{tcolorbox}

    \begin{Verbatim}[commandchars=\\\{\}]
Naive Bayes accuracy = 0.9955
    \end{Verbatim}

    \textbf{Expected Accuracy}:

\texttt{Naive\ Bayes\ accuracy\ =\ 0.9955}

    \begin{tcolorbox}[breakable, size=fbox, boxrule=1pt, pad at break*=1mm,colback=cellbackground, colframe=cellborder]
\prompt{In}{incolor}{17}{\boxspacing}
\begin{Verbatim}[commandchars=\\\{\}]
\PY{c+c1}{\PYZsh{} UNQ\PYZus{}C7 (UNIQUE CELL IDENTIFIER, DO NOT EDIT)}
\PY{c+c1}{\PYZsh{} Run this cell to test your function}
\PY{k}{for} \PY{n}{tweet} \PY{o+ow}{in} \PY{p}{[}\PY{l+s+s1}{\PYZsq{}}\PY{l+s+s1}{I am happy}\PY{l+s+s1}{\PYZsq{}}\PY{p}{,} \PY{l+s+s1}{\PYZsq{}}\PY{l+s+s1}{I am bad}\PY{l+s+s1}{\PYZsq{}}\PY{p}{,} \PY{l+s+s1}{\PYZsq{}}\PY{l+s+s1}{this movie should have been great.}\PY{l+s+s1}{\PYZsq{}}\PY{p}{,} \PY{l+s+s1}{\PYZsq{}}\PY{l+s+s1}{great}\PY{l+s+s1}{\PYZsq{}}\PY{p}{,} \PY{l+s+s1}{\PYZsq{}}\PY{l+s+s1}{great great}\PY{l+s+s1}{\PYZsq{}}\PY{p}{,} \PY{l+s+s1}{\PYZsq{}}\PY{l+s+s1}{great great great}\PY{l+s+s1}{\PYZsq{}}\PY{p}{,} \PY{l+s+s1}{\PYZsq{}}\PY{l+s+s1}{great great great great}\PY{l+s+s1}{\PYZsq{}}\PY{p}{]}\PY{p}{:}    
    \PY{n}{p} \PY{o}{=} \PY{n}{naive\PYZus{}bayes\PYZus{}predict}\PY{p}{(}\PY{n}{tweet}\PY{p}{,} \PY{n}{logprior}\PY{p}{,} \PY{n}{loglikelihood}\PY{p}{)}
    \PY{n+nb}{print}\PY{p}{(}\PY{l+s+sa}{f}\PY{l+s+s1}{\PYZsq{}}\PY{l+s+si}{\PYZob{}}\PY{n}{tweet}\PY{l+s+si}{\PYZcb{}}\PY{l+s+s1}{ \PYZhy{}\PYZgt{} }\PY{l+s+si}{\PYZob{}}\PY{n}{p}\PY{l+s+si}{:}\PY{l+s+s1}{.2f}\PY{l+s+si}{\PYZcb{}}\PY{l+s+s1}{\PYZsq{}}\PY{p}{)}
\end{Verbatim}
\end{tcolorbox}

    \begin{Verbatim}[commandchars=\\\{\}]
I am happy -> 2.14
I am bad -> -1.31
this movie should have been great. -> 2.12
great -> 2.13
great great -> 4.26
great great great -> 6.39
great great great great -> 8.52
    \end{Verbatim}

    \textbf{Expected Output}: - I am happy -\textgreater{} 2.14 - I am bad
-\textgreater{} -1.31 - this movie should have been great.
-\textgreater{} 2.12 - great -\textgreater{} 2.13 - great great
-\textgreater{} 4.26 - great great great -\textgreater{} 6.39 - great
great great great -\textgreater{} 8.52

    \begin{tcolorbox}[breakable, size=fbox, boxrule=1pt, pad at break*=1mm,colback=cellbackground, colframe=cellborder]
\prompt{In}{incolor}{18}{\boxspacing}
\begin{Verbatim}[commandchars=\\\{\}]
\PY{c+c1}{\PYZsh{} Feel free to check the sentiment of your own tweet below}
\PY{n}{my\PYZus{}tweet} \PY{o}{=} \PY{l+s+s1}{\PYZsq{}}\PY{l+s+s1}{you are bad :(}\PY{l+s+s1}{\PYZsq{}}
\PY{n}{naive\PYZus{}bayes\PYZus{}predict}\PY{p}{(}\PY{n}{my\PYZus{}tweet}\PY{p}{,} \PY{n}{logprior}\PY{p}{,} \PY{n}{loglikelihood}\PY{p}{)}
\end{Verbatim}
\end{tcolorbox}

            \begin{tcolorbox}[breakable, size=fbox, boxrule=.5pt, pad at break*=1mm, opacityfill=0]
\prompt{Out}{outcolor}{18}{\boxspacing}
\begin{Verbatim}[commandchars=\\\{\}]
-8.837351738825648
\end{Verbatim}
\end{tcolorbox}
        
    \begin{tcolorbox}[breakable, size=fbox, boxrule=1pt, pad at break*=1mm,colback=cellbackground, colframe=cellborder]
\prompt{In}{incolor}{19}{\boxspacing}
\begin{Verbatim}[commandchars=\\\{\}]
\PY{c+c1}{\PYZsh{} Test your function}
\PY{n}{w2\PYZus{}unittest}\PY{o}{.}\PY{n}{unittest\PYZus{}test\PYZus{}naive\PYZus{}bayes}\PY{p}{(}\PY{n}{test\PYZus{}naive\PYZus{}bayes}\PY{p}{,} \PY{n}{test\PYZus{}x}\PY{p}{,} \PY{n}{test\PYZus{}y}\PY{p}{)}
\end{Verbatim}
\end{tcolorbox}

    \begin{Verbatim}[commandchars=\\\{\}]
\textcolor{ansi-green-intense}{ All tests passed}
    \end{Verbatim}

    \hypertarget{part-4-filter-words-by-ratio-of-positive-to-negative-counts}{%
\section{Part 4: Filter words by Ratio of positive to negative
counts}\label{part-4-filter-words-by-ratio-of-positive-to-negative-counts}}

\begin{itemize}
\tightlist
\item
  Some words have more positive counts than others, and can be
  considered ``more positive''. Likewise, some words can be considered
  more negative than others.
\item
  One way for us to define the level of positiveness or negativeness,
  without calculating the log likelihood, is to compare the positive to
  negative frequency of the word.

  \begin{itemize}
  \tightlist
  \item
    Note that we can also use the log likelihood calculations to compare
    relative positivity or negativity of words.
  \end{itemize}
\item
  We can calculate the ratio of positive to negative frequencies of a
  word.
\item
  Once we're able to calculate these ratios, we can also filter a subset
  of words that have a minimum ratio of positivity / negativity or
  higher.
\item
  Similarly, we can also filter a subset of words that have a maximum
  ratio of positivity / negativity or lower (words that are at least as
  negative, or even more negative than a given threshold).
\end{itemize}

\hypertarget{implement-get_ratio}{%
\paragraph{Implement get\_ratio}\label{implement-get_ratio}}

\begin{itemize}
\tightlist
\item
  Given the freqs dictionary of words and a particular word, use
  \texttt{lookup(freqs,word,1)} to get the positive count of the word.
\item
  Similarly, use the \texttt{lookup} function to get the negative count
  of that word.
\item
  Calculate the ratio of positive divided by negative counts
\end{itemize}

\[ ratio = \frac{\text{pos_words} + 1}{\text{neg_words} + 1} \]

Where pos\_words and neg\_words correspond to the frequency of the words
in their respective classes.

Words

Positive word count

Negative Word Count

glad

41

2

arriv

57

4

:(

1

3663

:-(

0

378

    \begin{tcolorbox}[breakable, size=fbox, boxrule=1pt, pad at break*=1mm,colback=cellbackground, colframe=cellborder]
\prompt{In}{incolor}{20}{\boxspacing}
\begin{Verbatim}[commandchars=\\\{\}]
\PY{c+c1}{\PYZsh{} UNQ\PYZus{}C8 GRADED FUNCTION: get\PYZus{}ratio}

\PY{k}{def} \PY{n+nf}{get\PYZus{}ratio}\PY{p}{(}\PY{n}{freqs}\PY{p}{,} \PY{n}{word}\PY{p}{)}\PY{p}{:}
    \PY{l+s+sd}{\PYZsq{}\PYZsq{}\PYZsq{}}
\PY{l+s+sd}{    Input:}
\PY{l+s+sd}{        freqs: dictionary containing the words}

\PY{l+s+sd}{    Output: a dictionary with keys \PYZsq{}positive\PYZsq{}, \PYZsq{}negative\PYZsq{}, and \PYZsq{}ratio\PYZsq{}.}
\PY{l+s+sd}{        Example: \PYZob{}\PYZsq{}positive\PYZsq{}: 10, \PYZsq{}negative\PYZsq{}: 20, \PYZsq{}ratio\PYZsq{}: 0.5\PYZcb{}}
\PY{l+s+sd}{    \PYZsq{}\PYZsq{}\PYZsq{}}
    \PY{n}{pos\PYZus{}neg\PYZus{}ratio} \PY{o}{=} \PY{p}{\PYZob{}}\PY{l+s+s1}{\PYZsq{}}\PY{l+s+s1}{positive}\PY{l+s+s1}{\PYZsq{}}\PY{p}{:} \PY{l+m+mi}{0}\PY{p}{,} \PY{l+s+s1}{\PYZsq{}}\PY{l+s+s1}{negative}\PY{l+s+s1}{\PYZsq{}}\PY{p}{:} \PY{l+m+mi}{0}\PY{p}{,} \PY{l+s+s1}{\PYZsq{}}\PY{l+s+s1}{ratio}\PY{l+s+s1}{\PYZsq{}}\PY{p}{:} \PY{l+m+mf}{0.0}\PY{p}{\PYZcb{}}
    \PY{c+c1}{\PYZsh{}\PYZsh{}\PYZsh{} START CODE HERE (REPLACE INSTANCES OF \PYZsq{}None\PYZsq{} with your code) \PYZsh{}\PYZsh{}\PYZsh{}}
    \PY{c+c1}{\PYZsh{} use lookup() to find positive counts for the word (denoted by the integer 1)}
    \PY{n}{pos\PYZus{}neg\PYZus{}ratio}\PY{p}{[}\PY{l+s+s1}{\PYZsq{}}\PY{l+s+s1}{positive}\PY{l+s+s1}{\PYZsq{}}\PY{p}{]} \PY{o}{=} \PY{n}{lookup}\PY{p}{(}\PY{n}{freqs}\PY{p}{,} \PY{n}{word}\PY{p}{,} \PY{l+m+mi}{1}\PY{p}{)}

    \PY{c+c1}{\PYZsh{} use lookup() to find negative counts for the word (denoted by integer 0)}
    \PY{n}{pos\PYZus{}neg\PYZus{}ratio}\PY{p}{[}\PY{l+s+s1}{\PYZsq{}}\PY{l+s+s1}{negative}\PY{l+s+s1}{\PYZsq{}}\PY{p}{]} \PY{o}{=} \PY{n}{lookup}\PY{p}{(}\PY{n}{freqs}\PY{p}{,} \PY{n}{word}\PY{p}{,} \PY{l+m+mi}{0}\PY{p}{)}

    \PY{c+c1}{\PYZsh{} calculate the ratio of positive to negative counts for the word}
    \PY{n}{pos\PYZus{}neg\PYZus{}ratio}\PY{p}{[}\PY{l+s+s1}{\PYZsq{}}\PY{l+s+s1}{ratio}\PY{l+s+s1}{\PYZsq{}}\PY{p}{]} \PY{o}{=} \PY{p}{(}\PY{n}{pos\PYZus{}neg\PYZus{}ratio}\PY{p}{[}\PY{l+s+s1}{\PYZsq{}}\PY{l+s+s1}{positive}\PY{l+s+s1}{\PYZsq{}}\PY{p}{]} \PY{o}{+} \PY{l+m+mi}{1}\PY{p}{)} \PY{o}{/} \PY{p}{(}\PY{n}{pos\PYZus{}neg\PYZus{}ratio}\PY{p}{[}\PY{l+s+s1}{\PYZsq{}}\PY{l+s+s1}{negative}\PY{l+s+s1}{\PYZsq{}}\PY{p}{]} \PY{o}{+} \PY{l+m+mi}{1}\PY{p}{)}
    \PY{c+c1}{\PYZsh{}\PYZsh{}\PYZsh{} END CODE HERE \PYZsh{}\PYZsh{}\PYZsh{}}
    \PY{k}{return} \PY{n}{pos\PYZus{}neg\PYZus{}ratio}
\end{Verbatim}
\end{tcolorbox}

    \begin{tcolorbox}[breakable, size=fbox, boxrule=1pt, pad at break*=1mm,colback=cellbackground, colframe=cellborder]
\prompt{In}{incolor}{21}{\boxspacing}
\begin{Verbatim}[commandchars=\\\{\}]
\PY{n}{get\PYZus{}ratio}\PY{p}{(}\PY{n}{freqs}\PY{p}{,} \PY{l+s+s1}{\PYZsq{}}\PY{l+s+s1}{happi}\PY{l+s+s1}{\PYZsq{}}\PY{p}{)}
\end{Verbatim}
\end{tcolorbox}

            \begin{tcolorbox}[breakable, size=fbox, boxrule=.5pt, pad at break*=1mm, opacityfill=0]
\prompt{Out}{outcolor}{21}{\boxspacing}
\begin{Verbatim}[commandchars=\\\{\}]
\{'positive': 162, 'negative': 18, 'ratio': 8.578947368421053\}
\end{Verbatim}
\end{tcolorbox}
        
    \begin{tcolorbox}[breakable, size=fbox, boxrule=1pt, pad at break*=1mm,colback=cellbackground, colframe=cellborder]
\prompt{In}{incolor}{22}{\boxspacing}
\begin{Verbatim}[commandchars=\\\{\}]
\PY{c+c1}{\PYZsh{} Test your function}
\PY{n}{w2\PYZus{}unittest}\PY{o}{.}\PY{n}{test\PYZus{}get\PYZus{}ratio}\PY{p}{(}\PY{n}{get\PYZus{}ratio}\PY{p}{,} \PY{n}{freqs}\PY{p}{)}
\end{Verbatim}
\end{tcolorbox}

    \begin{Verbatim}[commandchars=\\\{\}]
\textcolor{ansi-green-intense}{ All tests passed}
    \end{Verbatim}

    \hypertarget{implement-get_words_by_thresholdfreqslabelthreshold}{%
\paragraph{Implement
get\_words\_by\_threshold(freqs,label,threshold)}\label{implement-get_words_by_thresholdfreqslabelthreshold}}

\begin{itemize}
\tightlist
\item
  If we set the label to 1, then we'll look for all words whose
  threshold of positive/negative is at least as high as that threshold,
  or higher.
\item
  If we set the label to 0, then we'll look for all words whose
  threshold of positive/negative is at most as low as the given
  threshold, or lower.
\item
  Use the \texttt{get\_ratio} function to get a dictionary containing
  the positive count, negative count, and the ratio of positive to
  negative counts.
\item
  Append the \texttt{get\_ratio} dictionary inside another dictinoary,
  where the key is the word, and the value is the dictionary
  \texttt{pos\_neg\_ratio} that is returned by the \texttt{get\_ratio}
  function. An example key-value pair would have this structure:
\end{itemize}

\begin{verbatim}
{'happi':
    {'positive': 10, 'negative': 20, 'ratio': 0.524}
}
\end{verbatim}

    \begin{tcolorbox}[breakable, size=fbox, boxrule=1pt, pad at break*=1mm,colback=cellbackground, colframe=cellborder]
\prompt{In}{incolor}{23}{\boxspacing}
\begin{Verbatim}[commandchars=\\\{\}]
\PY{c+c1}{\PYZsh{} UNQ\PYZus{}C9 (UNIQUE CELL IDENTIFIER, DO NOT EDIT)}
\PY{k}{def} \PY{n+nf}{get\PYZus{}words\PYZus{}by\PYZus{}threshold}\PY{p}{(}\PY{n}{freqs}\PY{p}{,} \PY{n}{label}\PY{p}{,} \PY{n}{threshold}\PY{p}{)}\PY{p}{:}
    \PY{l+s+sd}{\PYZsq{}\PYZsq{}\PYZsq{}}
\PY{l+s+sd}{    Input:}
\PY{l+s+sd}{        freqs: dictionary of words}
\PY{l+s+sd}{        label: 1 for positive, 0 for negative}
\PY{l+s+sd}{        threshold: ratio that will be used as the cutoff for including a word in the returned dictionary}
\PY{l+s+sd}{    Output:}
\PY{l+s+sd}{        word\PYZus{}set: dictionary containing the word and information on its positive count, negative count, and ratio of positive to negative counts.}
\PY{l+s+sd}{        example of a key value pair:}
\PY{l+s+sd}{        \PYZob{}\PYZsq{}happi\PYZsq{}:}
\PY{l+s+sd}{            \PYZob{}\PYZsq{}positive\PYZsq{}: 10, \PYZsq{}negative\PYZsq{}: 20, \PYZsq{}ratio\PYZsq{}: 0.5\PYZcb{}}
\PY{l+s+sd}{        \PYZcb{}}
\PY{l+s+sd}{    \PYZsq{}\PYZsq{}\PYZsq{}}
    \PY{n}{word\PYZus{}list} \PY{o}{=} \PY{p}{\PYZob{}}\PY{p}{\PYZcb{}}

    \PY{c+c1}{\PYZsh{}\PYZsh{}\PYZsh{} START CODE HERE (REPLACE INSTANCES OF \PYZsq{}None\PYZsq{} with your code) \PYZsh{}\PYZsh{}\PYZsh{}}
    \PY{k}{for} \PY{n}{key} \PY{o+ow}{in} \PY{n}{freqs}\PY{o}{.}\PY{n}{keys}\PY{p}{(}\PY{p}{)}\PY{p}{:}
        \PY{n}{word}\PY{p}{,} \PY{n}{\PYZus{}} \PY{o}{=} \PY{n}{key}

        \PY{c+c1}{\PYZsh{} get the positive/negative ratio for a word}
        \PY{n}{pos\PYZus{}neg\PYZus{}ratio} \PY{o}{=} \PY{n}{get\PYZus{}ratio}\PY{p}{(}\PY{n}{freqs}\PY{p}{,} \PY{n}{word}\PY{p}{)}

        \PY{c+c1}{\PYZsh{} if the label is 1 and the ratio is greater than or equal to the threshold...}
        \PY{k}{if} \PY{n}{label} \PY{o}{==} \PY{l+m+mi}{1} \PY{o+ow}{and} \PY{n}{pos\PYZus{}neg\PYZus{}ratio}\PY{p}{[}\PY{l+s+s1}{\PYZsq{}}\PY{l+s+s1}{ratio}\PY{l+s+s1}{\PYZsq{}}\PY{p}{]} \PY{o}{\PYZgt{}}\PY{o}{=} \PY{n}{threshold}\PY{p}{:}

            \PY{c+c1}{\PYZsh{} Add the pos\PYZus{}neg\PYZus{}ratio to the dictionary}
            \PY{n}{word\PYZus{}list}\PY{p}{[}\PY{n}{word}\PY{p}{]} \PY{o}{=} \PY{n}{pos\PYZus{}neg\PYZus{}ratio}

        \PY{c+c1}{\PYZsh{} If the label is 0 and the pos\PYZus{}neg\PYZus{}ratio is less than or equal to the threshold...}
        \PY{k}{elif} \PY{n}{label} \PY{o}{==} \PY{l+m+mi}{0} \PY{o+ow}{and} \PY{n}{pos\PYZus{}neg\PYZus{}ratio}\PY{p}{[}\PY{l+s+s1}{\PYZsq{}}\PY{l+s+s1}{ratio}\PY{l+s+s1}{\PYZsq{}}\PY{p}{]} \PY{o}{\PYZlt{}}\PY{o}{=} \PY{n}{threshold}\PY{p}{:}

            \PY{c+c1}{\PYZsh{} Add the pos\PYZus{}neg\PYZus{}ratio to the dictionary}
            \PY{n}{word\PYZus{}list}\PY{p}{[}\PY{n}{word}\PY{p}{]} \PY{o}{=} \PY{n}{pos\PYZus{}neg\PYZus{}ratio}

        \PY{c+c1}{\PYZsh{} otherwise, do not include this word in the list (do nothing)}

    \PY{c+c1}{\PYZsh{}\PYZsh{}\PYZsh{} END CODE HERE \PYZsh{}\PYZsh{}\PYZsh{}}
    \PY{k}{return} \PY{n}{word\PYZus{}list}
\end{Verbatim}
\end{tcolorbox}

    \begin{tcolorbox}[breakable, size=fbox, boxrule=1pt, pad at break*=1mm,colback=cellbackground, colframe=cellborder]
\prompt{In}{incolor}{24}{\boxspacing}
\begin{Verbatim}[commandchars=\\\{\}]
\PY{c+c1}{\PYZsh{} Test your function: find negative words at or below a threshold}
\PY{n}{get\PYZus{}words\PYZus{}by\PYZus{}threshold}\PY{p}{(}\PY{n}{freqs}\PY{p}{,} \PY{n}{label}\PY{o}{=}\PY{l+m+mi}{0}\PY{p}{,} \PY{n}{threshold}\PY{o}{=}\PY{l+m+mf}{0.05}\PY{p}{)}
\end{Verbatim}
\end{tcolorbox}

            \begin{tcolorbox}[breakable, size=fbox, boxrule=.5pt, pad at break*=1mm, opacityfill=0]
\prompt{Out}{outcolor}{24}{\boxspacing}
\begin{Verbatim}[commandchars=\\\{\}]
\{':(': \{'positive': 1, 'negative': 3675, 'ratio': 0.000544069640914037\},
 ':-(': \{'positive': 0, 'negative': 386, 'ratio': 0.002583979328165375\},
 'zayniscomingbackonjuli': \{'positive': 0, 'negative': 19, 'ratio': 0.05\},
 '26': \{'positive': 0, 'negative': 20, 'ratio': 0.047619047619047616\},
 '>:(': \{'positive': 0, 'negative': 43, 'ratio': 0.022727272727272728\},
 'lost': \{'positive': 0, 'negative': 19, 'ratio': 0.05\},
 '♛': \{'positive': 0, 'negative': 210, 'ratio': 0.004739336492890996\},
 '》': \{'positive': 0, 'negative': 210, 'ratio': 0.004739336492890996\},
 'beli̇ev': \{'positive': 0, 'negative': 35, 'ratio': 0.027777777777777776\},
 'wi̇ll': \{'positive': 0, 'negative': 35, 'ratio': 0.027777777777777776\},
 'justi̇n': \{'positive': 0, 'negative': 35, 'ratio': 0.027777777777777776\},
 'see': \{'positive': 0, 'negative': 35, 'ratio': 0.027777777777777776\},
 'me': \{'positive': 0, 'negative': 35, 'ratio': 0.027777777777777776\}\}
\end{Verbatim}
\end{tcolorbox}
        
    \begin{tcolorbox}[breakable, size=fbox, boxrule=1pt, pad at break*=1mm,colback=cellbackground, colframe=cellborder]
\prompt{In}{incolor}{25}{\boxspacing}
\begin{Verbatim}[commandchars=\\\{\}]
\PY{c+c1}{\PYZsh{} Test your function; find positive words at or above a threshold}
\PY{n}{get\PYZus{}words\PYZus{}by\PYZus{}threshold}\PY{p}{(}\PY{n}{freqs}\PY{p}{,} \PY{n}{label}\PY{o}{=}\PY{l+m+mi}{1}\PY{p}{,} \PY{n}{threshold}\PY{o}{=}\PY{l+m+mi}{10}\PY{p}{)}
\end{Verbatim}
\end{tcolorbox}

            \begin{tcolorbox}[breakable, size=fbox, boxrule=.5pt, pad at break*=1mm, opacityfill=0]
\prompt{Out}{outcolor}{25}{\boxspacing}
\begin{Verbatim}[commandchars=\\\{\}]
\{'followfriday': \{'positive': 23, 'negative': 0, 'ratio': 24.0\},
 'commun': \{'positive': 27, 'negative': 1, 'ratio': 14.0\},
 ':)': \{'positive': 2960, 'negative': 2, 'ratio': 987.0\},
 'flipkartfashionfriday': \{'positive': 16, 'negative': 0, 'ratio': 17.0\},
 ':D': \{'positive': 523, 'negative': 0, 'ratio': 524.0\},
 ':p': \{'positive': 104, 'negative': 0, 'ratio': 105.0\},
 'influenc': \{'positive': 16, 'negative': 0, 'ratio': 17.0\},
 ':-)': \{'positive': 552, 'negative': 0, 'ratio': 553.0\},
 "here'": \{'positive': 20, 'negative': 0, 'ratio': 21.0\},
 'youth': \{'positive': 14, 'negative': 0, 'ratio': 15.0\},
 'bam': \{'positive': 44, 'negative': 0, 'ratio': 45.0\},
 'warsaw': \{'positive': 44, 'negative': 0, 'ratio': 45.0\},
 'shout': \{'positive': 11, 'negative': 0, 'ratio': 12.0\},
 ';)': \{'positive': 22, 'negative': 0, 'ratio': 23.0\},
 'stat': \{'positive': 51, 'negative': 0, 'ratio': 52.0\},
 'arriv': \{'positive': 57, 'negative': 4, 'ratio': 11.6\},
 'glad': \{'positive': 41, 'negative': 2, 'ratio': 14.0\},
 'blog': \{'positive': 27, 'negative': 0, 'ratio': 28.0\},
 'fav': \{'positive': 11, 'negative': 0, 'ratio': 12.0\},
 'fantast': \{'positive': 9, 'negative': 0, 'ratio': 10.0\},
 'fback': \{'positive': 26, 'negative': 0, 'ratio': 27.0\},
 'pleasur': \{'positive': 10, 'negative': 0, 'ratio': 11.0\},
 '←': \{'positive': 9, 'negative': 0, 'ratio': 10.0\},
 'aqui': \{'positive': 9, 'negative': 0, 'ratio': 10.0\}\}
\end{Verbatim}
\end{tcolorbox}
        
    Notice the difference between the positive and negative ratios. Emojis
like :( and words like `me' tend to have a negative connotation. Other
words like glad, community, arrives, tend to be found in the positive
tweets.

    \begin{tcolorbox}[breakable, size=fbox, boxrule=1pt, pad at break*=1mm,colback=cellbackground, colframe=cellborder]
\prompt{In}{incolor}{26}{\boxspacing}
\begin{Verbatim}[commandchars=\\\{\}]
\PY{c+c1}{\PYZsh{} Test your function}
\PY{n}{w2\PYZus{}unittest}\PY{o}{.}\PY{n}{test\PYZus{}get\PYZus{}words\PYZus{}by\PYZus{}threshold}\PY{p}{(}\PY{n}{get\PYZus{}words\PYZus{}by\PYZus{}threshold}\PY{p}{,} \PY{n}{freqs}\PY{p}{)}
\end{Verbatim}
\end{tcolorbox}

    \begin{Verbatim}[commandchars=\\\{\}]
\textcolor{ansi-green-intense}{ All tests passed}
    \end{Verbatim}

    \hypertarget{part-5-error-analysis}{%
\section{Part 5: Error Analysis}\label{part-5-error-analysis}}

In this part you will see some tweets that your model missclassified.
Why do you think the missclassifications happened? Were there any
assumptions made by your naive bayes model?

    \begin{tcolorbox}[breakable, size=fbox, boxrule=1pt, pad at break*=1mm,colback=cellbackground, colframe=cellborder]
\prompt{In}{incolor}{27}{\boxspacing}
\begin{Verbatim}[commandchars=\\\{\}]
\PY{c+c1}{\PYZsh{} Some error analysis done for you}
\PY{n+nb}{print}\PY{p}{(}\PY{l+s+s1}{\PYZsq{}}\PY{l+s+s1}{Truth Predicted Tweet}\PY{l+s+s1}{\PYZsq{}}\PY{p}{)}
\PY{k}{for} \PY{n}{x}\PY{p}{,} \PY{n}{y} \PY{o+ow}{in} \PY{n+nb}{zip}\PY{p}{(}\PY{n}{test\PYZus{}x}\PY{p}{,} \PY{n}{test\PYZus{}y}\PY{p}{)}\PY{p}{:}
    \PY{n}{y\PYZus{}hat} \PY{o}{=} \PY{n}{naive\PYZus{}bayes\PYZus{}predict}\PY{p}{(}\PY{n}{x}\PY{p}{,} \PY{n}{logprior}\PY{p}{,} \PY{n}{loglikelihood}\PY{p}{)}
    \PY{k}{if} \PY{n}{y} \PY{o}{!=} \PY{p}{(}\PY{n}{np}\PY{o}{.}\PY{n}{sign}\PY{p}{(}\PY{n}{y\PYZus{}hat}\PY{p}{)} \PY{o}{\PYZgt{}} \PY{l+m+mi}{0}\PY{p}{)}\PY{p}{:}
        \PY{n+nb}{print}\PY{p}{(}\PY{l+s+s1}{\PYZsq{}}\PY{l+s+si}{\PYZpc{}d}\PY{l+s+se}{\PYZbs{}t}\PY{l+s+si}{\PYZpc{}0.2f}\PY{l+s+se}{\PYZbs{}t}\PY{l+s+si}{\PYZpc{}s}\PY{l+s+s1}{\PYZsq{}} \PY{o}{\PYZpc{}} \PY{p}{(}\PY{n}{y}\PY{p}{,} \PY{n}{np}\PY{o}{.}\PY{n}{sign}\PY{p}{(}\PY{n}{y\PYZus{}hat}\PY{p}{)} \PY{o}{\PYZgt{}} \PY{l+m+mi}{0}\PY{p}{,} \PY{l+s+s1}{\PYZsq{}}\PY{l+s+s1}{ }\PY{l+s+s1}{\PYZsq{}}\PY{o}{.}\PY{n}{join}\PY{p}{(}
            \PY{n}{process\PYZus{}tweet}\PY{p}{(}\PY{n}{x}\PY{p}{)}\PY{p}{)}\PY{o}{.}\PY{n}{encode}\PY{p}{(}\PY{l+s+s1}{\PYZsq{}}\PY{l+s+s1}{ascii}\PY{l+s+s1}{\PYZsq{}}\PY{p}{,} \PY{l+s+s1}{\PYZsq{}}\PY{l+s+s1}{ignore}\PY{l+s+s1}{\PYZsq{}}\PY{p}{)}\PY{p}{)}\PY{p}{)}
\end{Verbatim}
\end{tcolorbox}

    \begin{Verbatim}[commandchars=\\\{\}]
Truth Predicted Tweet
1       0.00    b'truli later move know queen bee upward bound movingonup'
1       0.00    b'new report talk burn calori cold work harder warm feel better
weather :p'
1       0.00    b'harri niall 94 harri born ik stupid wanna chang :D'
1       0.00    b'park get sunlight'
1       0.00    b'uff itna miss karhi thi ap :p'
0       1.00    b'hello info possibl interest jonatha close join beti :( great'
0       1.00    b'u prob fun david'
0       1.00    b'pat jay'
0       1.00    b'sr financi analyst expedia inc bellevu wa financ expediajob
job job hire'
    \end{Verbatim}

    \hypertarget{part-6-predict-with-your-own-tweet}{%
\section{Part 6: Predict with your own
tweet}\label{part-6-predict-with-your-own-tweet}}

In this part you can predict the sentiment of your own tweet.

    \begin{tcolorbox}[breakable, size=fbox, boxrule=1pt, pad at break*=1mm,colback=cellbackground, colframe=cellborder]
\prompt{In}{incolor}{28}{\boxspacing}
\begin{Verbatim}[commandchars=\\\{\}]
\PY{c+c1}{\PYZsh{} Test with your own tweet \PYZhy{} feel free to modify `my\PYZus{}tweet`}
\PY{n}{my\PYZus{}tweet} \PY{o}{=} \PY{l+s+s1}{\PYZsq{}}\PY{l+s+s1}{I am happy because I am learning :)}\PY{l+s+s1}{\PYZsq{}}

\PY{n}{p} \PY{o}{=} \PY{n}{naive\PYZus{}bayes\PYZus{}predict}\PY{p}{(}\PY{n}{my\PYZus{}tweet}\PY{p}{,} \PY{n}{logprior}\PY{p}{,} \PY{n}{loglikelihood}\PY{p}{)}
\PY{n+nb}{print}\PY{p}{(}\PY{n}{p}\PY{p}{)}
\end{Verbatim}
\end{tcolorbox}

    \begin{Verbatim}[commandchars=\\\{\}]
9.571143871339594
    \end{Verbatim}

    Congratulations on completing this assignment. See you next week!


    % Add a bibliography block to the postdoc
    
    
    
\end{document}
